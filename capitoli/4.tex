% !TEX encoding = UTF-8
% !TEX TS-program = pdflatex
% !TEX root = ../tesi.tex
% !TEX spellcheck = it-IT

%************************************************
\chapter{\texorpdfstring{Learning strutturale}{Apprendimento strutturale}}
\label{cap:ctbn-structural-learning}
%************************************************
\lipsum[1]
% TODO: 5.2 da Nodelman2007, paper Nodelman2002

% Several considerations concerning the exploitation of the structure of the graph G of the CTBNC in the case where the classification task is performed on a fully observed J-evidence-stream can be made. In such an evidential setting, the only unobserved random variable is the class variable Y; thus it is possible to fruitfully and conditionally exploit independence relationships between random variables as it happens in ordinary BNs

% This is a form of unsupervised learning, in the sense that the learner does not distinguish the class variable from the attribute
% variables in the data. The objective is to induce a network (or a set of networks) that “best describes” the probability distribution over the training data. This optimization process is implemented in practice by using heuristic search techniques to find the best candidate over the space of possible networks. The search process relies on a scoring function that assesses the merits of each candidate network.

\section{Score}
\label{sec:ctbn-score}
...

\section{Ricerca della struttura}
\label{sec:ctbn-graph-search}
...

\subsection{Hill Climbing}
\label{sec:hc}
...
