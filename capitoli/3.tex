% !TEX encoding = UTF-8
% !TEX TS-program = pdflatex
% !TEX root = ../tesi.tex
% !TEX spellcheck = it-IT

%************************************************
\chapter{Classificazione}
\label{cap:ctbnc}
%************************************************
In questo capitolo viene introdotta una classe di modelli, che prende il nome di \acf{CTBNC}, il cui scopo è la classificazione supervisionata di traiettorie multivariate di variabili discrete a tempo continuo e spazio degli stati discreto. Si descrivono due istanze di tale classe: i classificatori \acf{CTNB} nella \autoref{sec:learning-ctnb} e i classificatori \acf{CTTANB} nella \autoref{sec:learning-cttanb}, per le quali si affronta il processo di apprendimento in caso di dati completi.

Infine, nella \autoref{sec:inference-ctbnc}, si presenta un algoritmo di inferenza esatta per la classe dei \acs{CTBNC}.

\section{Apprendimento}
Ciao!

% \digraph
% [scale=0.7]{prova}
% {
%     main -> parse -> execute;
%     main -> init;
%     main -> cleanup;
%     execute -> make_string;
%     execute -> printf
%     init -> make_string;
%     main -> printf;
%     execute -> compare;
% }

\subsection{Na\"ive Bayes}\label{sec:learning-ctnb}
...

\subsection{Tree Augumented Na\"ive Bayes}\label{sec:learning-cttanb}
...

\section{Inferenza}\label{sec:inference-ctbnc}
...

\subsection{Na\"ive Bayes}\label{sec:inference-ctnb}
...


% TODO
% They implement a trade-off between computational complexity and classification accuracy.
% The performance of the continuous time naive Bayes classifier is assessed in the case where real-time feedback to neurological patients undergoing motor rehabilitation must be provided
