% !TEX encoding = UTF-8
% !TEX TS-program = pdflatex
% !TEX root = ../tesi.tex
% !TEX spellcheck = it-IT

%************************************************
\chapter{CTBN}
\label{cap:ctbn}
%************************************************

In questo capitolo si introducono i concetti fondamentali delle \ctbn{}, un framework utile a modellare i processi stocastici relativi a uno spazio degli stati che evolve nel tempo.
Tuttavia, prima di affrontare tale argomento si ritiene indispensabile presentare i suoi fondamenti matematici: le reti Bayesiane e i processi di Markov, trattati entrambi nella \autoref{sec:fondamenti}.


\section{Fondamenti}
\label{sec:fondamenti}

Come accennato, le \ctbn{} sono un modello che sviluppa e integra idee prese in prestito da altre teorie.
\begin{description}
\item[Reti Bayesiane] \hfill \\
Le \ctbn{} utilizzano una rappresentazione strutturata dello spazio degli stati propria della teoria delle \bns{}, ereditandone perciò gli aspetti chiave (\ie{} indipendenza condizionale) nonché l'insieme delle tecniche algoritmiche efficienti per l'apprendimento e l'inferenza.
\item[Processi di Markov] \hfill \\
Le \ctbn{} descrivono la dinamica evolutiva delle variabili casuali da cui sono costituite tramite dei processi di Markov omogenei.
\end{description} 

\subsection{Reti Bayesiane}
\label{sec:bns}

\subsection{Processi di Markov}
\label{sec:mps}
omogenei e non. stati finiti. tempo continuo.


\section{Rappresentazione}


\section{Inferenza}


\section{Apprendimento}

