% !TEX encoding = UTF-8
% !TEX TS-program = pdflatex
% !TEX root = ../tesi.tex
% !TEX spellcheck = it-IT

%************************************************
\chapter{\texorpdfstring{CTBN}{\ctbn{}}}
\label{cap:ctbn}
%************************************************
\acresetall
In questo capitolo si introducono i concetti fondamentali delle \ac{CTBN}, un framework utile a modellare i processi stocastici relativi a uno spazio degli stati che evolve nel tempo.
Tuttavia, prima di affrontare tale argomento si ritiene indispensabile presentare i suoi fondamenti matematici: le \ac{BN} e i \mprocess{}, trattati entrambi nella \autoref{sec:fondamenti}.

%************************************************
\section{Fondamenti}
\label{sec:fondamenti}
Come accennato, le \acl{CTBN} sviluppano e integrano concetti e idee provenienti da altre teorie afferenti l'area statistica e del machine learning. Al fine di conferire alla discussione sulle \acs{CTBN} un quadro iniziale completo ed esauriente, si presentano quindi gli aspetti di maggior rilievo di tali argomenti.
\begin{description}
\item[\bn{}] \hfill \\
Le \acl{CTBN} utilizzano una rappresentazione strutturata dello spazio degli stati propria della teoria delle \acl{BN}. Ne ereditano perciò gli aspetti chiave (\eg{} indipendenza condizionale) nonché l'insieme delle tecniche algoritmiche efficienti per l'apprendimento e l'inferenza.
\item[\mprocess{}] \hfill \\
Le \acl{CTBN} descrivono la dinamica evolutiva delle variabili casuali da cui sono costituite tramite un \omprocess{} globale, costituito da un insieme di \cmprocess{}.
\end{description} 

\subsection{\bn{}}
\label{sec:bns}
Una \acl{BN} è un modello grafico probabilistico costituito da un \acf{DAG}.\footnote{Un grafo aciclico diretto (anche detto grafo orientato aciclico o digrafo aciclico) è un tipo di grafo che non presenta cicli diretti: comunque si scelga un vertice non è possibile tornare ad esso percorrendo gli archi del grafo.} I nodi di tale grafo rappresentano un insieme di variabili casuali mentre gli archi rappresentano le dipendenze condizionali fra esse. Una \acs{BN} rappresenta la distribuzione della probabilità congiunta del suo insieme di variabili casuali. Per tale motivo questo modello statistico è comunemente utilizzato per modellare le relazioni probabilistiche fra eventi.

chain rule
bayes rule

Bayesian networks are used for modelling knowledge in computational biology and bioinformatics (gene regulatory networks, protein structure, gene expression analysis,[15] learning epistasis from GWAS data sets [16]) medicine,[17] biomonitoring,[18] document classification, information retrieval,[19] semantic search,[20] image processing, data fusion, decision support systems,[21] engineering, gaming and law.[22][23][24]

Le reti Bayesiane (BN) sono modelli grafici probabilistici 
per la rappresentazione e l’analisi di modelli che 
coinvolgono incertezza. 
- Sono usate in svariate applicazioni, ad esempio come 
supporto alle decisioni, in sistemi diagnostici, in data 
mining, bioinformatica… 

\subsubsection{Rappresentazione}

\begin{definizione}[\acl{BN}]
Una \acl{BN} $\conceptsym{B}$ è una coppia $\conceptsym{B}=(\conceptsym{G}, \set{P})$ costituita da:
\begin{itemize}
    \item $\conceptsym{G}=(\set{V}(\conceptsym{G}),\set{A}(\conceptsym{G}))$, un \acl{DAG} dove:
    \begin{itemize}
        \item $\set{V}(\conceptsym{G})=\{\setel{V_1}, \dotsc, \setel{V_n}\}$ è l'insieme dei vertici
        \item $\set{A}(\conceptsym{G})\subseteq\set{V}(\conceptsym{G})\times\set{V}(\conceptsym{G})$ è l'insieme degli archi fra i vertici
    \end{itemize}
    \item $\set{P}$, una distribuzione di probabilità congiunta definita sulle variabili casuali $\set{X}_{\set{V}(\conceptsym{G})}$ a cui corrispondono i vertici $\set{V}(\conceptsym{G})$.
\end{itemize}
\end{definizione}

\begin{definizione}[Indipendenza]
\end{definizione} 

\begin{definizione}[Indipendenza condizionale]
\end{definizione} 

\subsubsection{Apprendimento}

\subsubsection{Inferenza}

\subsection{\mprocess{}}
\label{sec:mps}

\begin{definizione}[Assunzione di Markov]
\end{definizione} 

\begin{definizione}[\upcase\omprocess{}]
\end{definizione} 

\begin{definizione}[Matrice di intensità]
\end{definizione} 

\begin{definizione}[\upcase\cmprocess{}]
\end{definizione} 

\begin{definizione}[Matrice di intensità condizionata]
\end{definizione} 

%************************************************
\section{Rappresentazione}
\label{sec:rappresentazione}

Una \acl{CTBN} è un modello grafico in cui ogni nodo è associato con una variabile casuale i cui stati evolvono nel tempo continuo. Si assume perciò che tali dinamiche evolutive siano governate e dipendano dal valore che gli stati dei nodi padre assumono \cite{Stella2012}.
Una \acs{CTBN} è composta principalmente da due componenti:
\begin{itemize}
    \item una distribuzione di probabilità iniziale
    \item le dinamiche che regolano l'evoluzione nel tempo continuo della distribuzione di probabilità
\end{itemize}
Più formalmente si definisce:
\begin{definizione}[\acl{CTBN}]
Dato un insieme $\set{X}$ di variabili casuali $\setel{X_1}, \setel{X_2}, \dotsc, \setel{X_N}$ dove ogni $\setel{X_n}$ ha uno spazio degli stati finito $val(\setel{X_N})=\{ \vectel{x_1}, \dotsc, \vectel{x_J} \}$, una \acs{CTBN} $\conceptsym{N}$ su $\set{X}$ consiste di:
\begin{itemize}
    \item una distribuzione di probabilità iniziale $\priorsign{\set{X}}$ specificata come una \bn{} $\conceptsym{B}$ su $\set{X}$
    \item un modello di transizione continuo, specificato da:
    \begin{itemize}
        \item un grafo $\conceptsym{G}$, orientato e non necessariamente aciclico, composto dai nodi $\setel{X_1}, \setel{X_2}, \dotsc, \setel{X_N}$, ognuno dei quali possiede un insieme di genitori (possibilmente vuoto) denotato da $pa(\setel{X_n})$
        \item una matrice di intensità condizionale $\cimsign{\setel{X_n}}$ per ogni nodo $\setel{X_n} \in \set{X}$.
    \end{itemize}
\end{itemize}
\end{definizione}
\begin{nota}
Si noti che, diversamente dalle \acl{BN}, nelle \acl{CTBN} gli archi fra i nodi rappresentano le relazioni temporali fra essi. Tali relazioni codificano le dinamiche evolutive dei nodi, ognuna delle quali è espressa condizionatamente alla dinamica evolutiva degli stati dei suoi nodi genitori. Per tale motivo è possibile che la componente $\conceptsym{G}$ del modello di transizione continuo contenga dei cicli. Tra l'altro, come vedremo nel prosieguo, la mancanza di tale vincolo di aciclicità porta a notevoli vantaggi computazionali relativamente all'apprendimento della struttura di una \acs{CTBN} dai dati. 
\end{nota}

%************************************************
\section{Apprendimento}
\label{sec:apprendimento}

%************************************************
\section{Inferenza}
\label{sec:inferenza}


% TODO: apprendimento strutturale?
% TODO: inserire immagini esemplificative delle ctbn
