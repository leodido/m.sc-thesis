% !TEX encoding = UTF-8
% !TEX TS-program = pdflatex
% !TEX root = ../tesi.tex
% !TEX spellcheck = it-IT

%%%%%%%%%%%%%%%%%%%%%%%%%%%%%%%%%%%%%%%%%%%%%%%%%%%%%%%%%%%%%%%%%%%%%%%%%%%%%%%%
\chapter{\texorpdfstring{CTBN}{\ctbn{}}}
\label{cap:ctbn}
%%%%%%%%%%%%%%%%%%%%%%%%%%%%%%%%%%%%%%%%%%%%%%%%%%%%%%%%%%%%%%%%%%%%%%%%%%%%%%%%
\acresetall
In questo capitolo si introducono i concetti fondamentali delle \ac{CTBN}, un framework utile a modellare i processi stocastici relativi a uno spazio degli stati che evolve nel tempo.
Tuttavia, prima di affrontare tale argomento si ritiene indispensabile presentare i suoi fondamenti matematici: le \ac{BN} e i \mprocess{}, trattati entrambi nella \autoref{sec:fondamenti}.

%%%%%%%%%%%%%%%%%%%%%%%%%%%%%%%%%%%%%%%%%%%%%%%%%%%%%%%%%%%%%%%%%%%%%%%%%%%%%%%%
\section{Fondamenti}
\label{sec:fondamenti}
%%%%%%%%%%%%%%%%%%%%%%%%%%%%%%%%%%%%%%%%%%%%%%%%%%%%%%%%%%%%%%%%%%%%%%%%%%%%%%%%
Come accennato, le \acl{CTBN} sviluppano e integrano concetti e idee provenienti da altre teorie afferenti l'area statistica e del machine learning. Al fine di conferire alla discussione sulle \acs{CTBN} un quadro iniziale completo ed esauriente, si presentano quindi gli aspetti di maggior rilievo di tali argomenti.
\begin{description}
\item[\bn{}] \hfill \\
Le \acl{CTBN} utilizzano una rappresentazione strutturata dello spazio degli stati propria della teoria delle \acl{BN}. Ne ereditano perciò gli aspetti chiave (\eg{} indipendenza condizionale) nonché l'insieme delle tecniche algoritmiche efficienti per l'apprendimento e l'inferenza.
\item[\mprocess{}] \hfill \\
\label{sec:fondamenti-mp}
Le \acl{CTBN} descrivono la dinamica evolutiva delle variabili casuali da cui sono costituite tramite un \homm\mprocess globale, costituito da un insieme di \conm\mprocess{}.
\end{description} 

\subsection{\bn{}}
\label{sec:bn}
Una \acl{BN} è un modello grafico probabilistico costituito da un \acf{DAG}.\footnote{Un grafo aciclico diretto (anche detto grafo orientato aciclico o digrafo aciclico) è un tipo di grafo che non presenta cicli diretti: comunque si scelga un vertice non è possibile tornare ad esso percorrendo gli archi del grafo.} I nodi di tale grafo rappresentano un insieme di variabili casuali mentre gli archi rappresentano le dipendenze (e le indipendenze) condizionali fra esse. 
Una \acs{BN} rappresenta la distribuzione di probabilità congiunta del suo insieme di variabili casuali tramite la distribuzione di probabilità condizionale di ognuna di essa \eqref{eq:bn-chain-rule}.
Le \acs{BN} sono quindi modelli grafico probabilistici con cui è possibile modellare in modo probabilistico le relazioni causali dirette fra eventi. Perciò esse risultano molto utili nella rappresentazione e analisi di modelli del mondo reale che coinvolgono incertezza. Sono infatti usate in svariate applicazioni di supporto alle decisioni, bioinformatica, biologia computazionale, data mining, information retrieval e classificazione.

\subsubsection{Rappresentazione}
Di seguito si dà la definizione formale delle \acl{BN} e si introducono i loro aspetti basilari.

\begin{definizione}[\acl{BN}]
\label{defn:bn}
Una \acl{BN} $\conceptsym{B}$ è una coppia $\conceptsym{B}=(\conceptsym{G},\theta_{\conceptsym{G}})$ costituita da:
\begin{itemize}
    \item $\conceptsym{G}=(\set{V}(\conceptsym{G}),\set{A}(\conceptsym{G}))$, un \acl{DAG} dove:
    \begin{itemize}
        \item $\set{V}(\conceptsym{G})=\{\setel{V_1}, \dotsc, \setel{V_n}\}$ è l'insieme dei nodi, ognuno dei quali è associato con una \acf{CPD}\footnote{Nel caso di variabili causali discrete, le \acs{CPD} sono rappresentabili come delle tabelle che contengono i valori di probabilità di un nodo in funzione di tutte le possibili configurazioni dei nodi genitori (cioè l'insieme dei nodi da cui parte un arco che punta al nodo di interesse).}
        \label{defn:bn-markov-assumption}
        \item ogni nodo è condizionalmente indipendente (\ref{defn:ic}) dai suoi non--discendenti dati i suoi nodi genitori
        \item $\set{A}(\conceptsym{G})\subseteq\set{V}(\conceptsym{G})\times\set{V}(\conceptsym{G})$ è l'insieme degli archi fra i nodi $\set{V}(\conceptsym{G})$
    \end{itemize}
    \item $\theta_{\conceptsym{G}}$, insieme delle \acs{CPD} dei nodi che specifica $\set{P}_{\conceptsym{B}}$, la distribuzione di probabilità congiunta delle variabili casuali $\set{X}_{\set{V}(\conceptsym{G})}$ a cui corrispondono i nodi $\set{V}(\conceptsym{G})$.
\end{itemize}
\end{definizione}

La \acs{CPD} di ogni variabile casuale $\setel{X_i}\in\set{X}_{\set{V}(\conceptsym{G})}$ esprime i suoi valori di probabilità in funzione dei valori assunti da $pa(\setel{X_i})$, notazione con cui denotiamo l'\emph{insieme dei nodi genitori} per ogni nodo o variabile casuale.

Mentre un arco da un nodo genitore verso un nodo figlio di $\conceptsym{G}$ rappresenta una dipendenza condizionale fra le corrispettive variabili casuali, i nodi non connessi, invece, rappresentano variabili casuali condizionalmente indipendenti dagli altri nodi (per quanto riguarda il concetto di \emph{indipendenza condizionale} si rimanda invece a \ref{defn:ic}).

Prima di procedere con la discussione si introduce la Chain Rule, regola molto utile nel caso delle \acs{BN}.

\begin{teorema}[Chain Rule]
Dato un insieme di variabili casuali e una distribuzione di probabilità congiunta definita su di esse è possibile calcolare qualsiasi elemento di tale distribuzione tramite le distribuzioni di probabilità condizionale delle variabili casuali~\cite{Russel2003}.
\end{teorema}

Perciò, dato un insieme di variabili casuali $\setel{A_1},\dotsc,\setel{A_n}$ è possibile calcolare il valore di tale membro della distribuzione di probabilità congiunta applicando la definizione di probabilità condizionale:
\[
\set{P}(\setel{A_1},\dotsc,\setel{A_n})=\set{P}(\setel{A_n}\,|\,\setel{A_{n-1}},\dotsc,\setel{A_1})\cdot\set{P}(\setel{A_{n-1}},\dotsc,\setel{A_1})
\]
Ripetendo tale processo per ogni termine finale si ottiene:
\begin{equation}
\label{eq:chain-rule}
\set{P}(\bigcap_{k=1}^{n}\setel{A_k})=\prod_{k=1}^{n}\set{P}(\setel{A_k}\,|\,\bigcap_{j=1}^{k-1}\setel{A_j})
\end{equation}

Applicando la \eqref{eq:chain-rule} alle \acl{BN} diciamo che la distribuzione di probabilità congiunta $\set{P}_{\conceptsym{B}}$ si \emph{fattorizza} rispetto al grafo $\conceptsym{G}$ se è possibile scrivere:
\begin{equation}
\label{eq:bn-chain-rule}
\set{P}_{\conceptsym{B}}(\setel{X_1}, \dotsc, \setel{X_n})=\prod_{i=1}^{n}\set{P}(\setel{X_i}\,|\,pa(\setel{X_i}))
\end{equation}
L'equazione \eqref{eq:bn-chain-rule} esprime quindi la \emph{proprietà di fattorizzazione} della distribuzione congiunta del modello grafico, ed è ciò che permette di descriverla efficientemente in funzione delle distribuzioni condizionali dei nodi. Si noti inoltre, che le \acl{BN} richiedono che la loro componente $\conceptsym{G}$ non contenga cicli (si veda \acs{DAG} in \ref{defn:bn}) affinché possano rispettare tale proprietà.

Poiché, come detto, una \acl{BN} stabilisce che ogni nodo, dati i suoi immediati ascendenti (i.e. nodi genitori), è \emph{condizionalmente indipendente} da ogni altro nodo che non sia un suo discendente, di seguito introduciamo tale concetto formalmente.
\begin{definizione}[Indipendenza condizionale]\label{defn:ic}
Un evento $\setel{A}$ è \emph{condizionalmente indipendente} da un evento $\setel{B}$, data l'evidenza su un evento $\setel{C}$, qualora la conoscenza di $\setel{B}$ non apporta alcuna variazione alla probabilità di $\setel{A}$ rispetto a quella conseguente alla conoscenza di $\setel{C}$.
Formalmente, ciò significa che:
\[
\set{P}(\setel{A},\setel{B}\,|\,\setel{C})=\set{P}(\setel{A}\,|\,\setel{B},\setel{C})\cdot\set{P}(\setel{B}\,|\,\setel{C})=\set{P}(\setel{A}\,|\,\setel{C})\cdot\set{P}(\setel{B}\,|\,\setel{C})
\]
Da cui segue che:
\[
\setel{A}\perp\setel{B}\,|\,\setel{C}\iff\set{P}(\setel{A}\,|\,\setel{B},\setel{C})=\set{P}(\setel{A}\,|\,\setel{C})
\]
\end{definizione}
In termini non formali, supponendo di essere nel caso della definizione, cioè di avere una variabile casuale $\setel{A}$ \emph{condizionalmente indipendente} da $\setel{B}$ dato $\setel{C}$, ciò significa che è possibile ignorare $\setel{B}$ poiché essa non ha alcun riflesso sulla distribuzione condizionale di $\setel{A}$.

Si noti che il concetto appena espresso gioca quindi un ruolo importante per i modelli probabilistici, quali sono le \acl{BN}: semplifica la struttura del modello e i calcoli richiesti per l'inferenza e l'apprendimento del modello. Le \acl{BN} ereditano questi benefici dell'indipendenza condizionale come conseguenza della loro definizione (si veda \ref{defn:bn-markov-assumption}): la distribuzione condizionale di ogni variabile casuale $\setel{X_i}$ dipende solo ed esclusivamente dal valore dei suoi genitori $pa(\setel{X_i})$ mentre ignora completamente le variabili casuali associate a nodi non--discendenti di $\setel{X_i}$. In breve:

$\forall$ $\setel{X_i}\in\set{X}_{\set{V}(\conceptsym{G})}$:
\[
\set{P}(\setel{X_i}\,|\,\setel{E},pa(\setel{X_i}))=\set{P}(\setel{X_i}\,|\,pa(\setel{X_i}))\;\forall\:\setel{E}\in nd(\setel{X_i})
\]
dove $nd(\setel{X_i})$ è l'insieme dei nodi non--discendenti (ed $\setel{E}$ è una variabile casuale o un insieme di variabili casuali ad essi associati).
In base a ciò si dice quindi che le \acl{BN} rispettano l'\emph{assunzione locale di Markov}.

\subsubsection{Apprendimento e Inferenza}\acresetall
In questa sezione si descrivono brevemente, e solo a scopo introduttivo, i processi di apprendimento e inferenza sulle \acl{BN}.

Il problema dell'apprendimento per le \acl{BN} si divide principalmente in due casi:
\begin{itemize}
    \item apprendere le \acs{CPD}, nota la struttura 
    \item apprendere sia le \acs{CPD}, sia la struttura (incognita).
\end{itemize}
In entrambi i casi è di grande aiuto la rappresentazione efficiente delle \acl{BN} che, tramite la \emph{fattorizzazione} della distribuzione di probabilità congiunta, permette di rappresentarla in modo compatto \eqref{eq:bn-chain-rule} riducendo notevolmente il numero di parametri da calcolare.

Come detto, per specificare completamente una \acl{BN} è necessario rappresentare completamente la distribuzione di probabilità congiunta delle sue variabili tramite la \acl{CPD} di ognuna di esse. In generale, tali distribuzioni condizionali possono avere una qualsiasi forma anche se, al fine di semplificare i calcoli, è comune utilizzare distribuzioni discrete o Gaussiane per modellarle. \`E frequente la situazione in cui queste distribuzioni condizionali includono parametri sconosciuti che è necessario stimare dai dati. In tali casi solitamente si procede tramite l'algoritmo di \acf{EM}, il quale alterna il calcolo dei valori attesi delle variabili casuali non osservate condizionalmente ai dati osservati con la massimizzazione della likelihood. Tale approccio generalmente converge ai valori di massima probabilità a posteriori per i parametri. Esistono comunque una varietà di altri approcci possibili (\eg{} trattare i parametri come variabili casuali sconosciute addizionali) per l'\emph{apprendimento dei parametri} che tuttavia non sono argomento di questo lavoro di tesi.

Si noti che le \acl{BN} non sono solamente un \emph{modello discriminativo ma anche generativo} poiché possono essere utilizzate per soddisfare query arbitrarie, cioè per effettuare \emph{inferenza probabilistica}: calcolare la distribuzione a posteriori di un insieme di variabili casuali data l'osservazione (evidenza) di altre (sfruttando il \emph{teorema di Bayes}). In letteratura sono stati esplorati molti metodi di \emph{inferenza esatta}, quali ad esempio l'eliminazione tramite integrazione o somma delle variabili non osservate che non fanno parte della query probabilistica o il metodo clique tree proprogation. Questi metodi, come gli altri presenti in letterature, sono esponenziali rispetto al tree-width del grafo. Per quanto riguarda invece gli algoritmi di \emph{inferenza approssimata} si citano due tra i più comuni: l'importance sampling e la simulazione \acf{MCMC}.

Quando si dispone del grafo sottostante una \acs{BN}, aldilà dell'osservabilità o meno delle sue variabili casuali, è possibile quindi usare tale \acs{BN} per effettuare inferenza. Nel caso in cui invece non si disponga della struttura di una \acs{BN} si presenta innanzitutto il problema dell'\emph{apprendimento strutturale}. Per apprendere automaticamente il grafo di una \acl{BN} sono stati sviluppati principalmente due famiglie di algoritmi:
\begin{itemize}
    \item algoritmi basati sulla ricerca dello scheletro del grafo dal quale, tramite le indipendenze condizionali osservate, si deducono successivamente le direzioni degli archi
    \item algoritmi che utilizzano tecniche di ricerca ottimizzata.
\end{itemize}
Gli algoritmi facenti parte della seconda famiglia utilizzando una strategia di ricerca (\eg{} Hill Climbing, Best First Search, Simulated Annealing) e una funzione di scoring. Solitamente la funzione di scoring utilizza la probabilità a posteriori della struttura in esame, dato l'insieme dei dati di apprendimento (\ie{} training set). Tuttavia, per quanto questi algoritmi siano utilizzati molto frequentemente, essi sono super esponenziali rispetto al numero di nodi della struttura del grafo. Inoltre, qualora si utilizzi una strategia di ricerca locale, è possibile che l'algoritmo restituisca come risultato un minimo locale (per evitare questa situazione si ricorre spesso a metodi di ricerca globale quali il \acs{MCMC}). Si fa notare che è possibile ridurre il tempo necessario richiesto per l'apprendimento strutturale fissando un numero massimo di genitori candidati e cercando esaustivamente in insiemi di tale cardinalità una struttura che massimizzi l'informazione mutua fra variabili.

\subsection{\mprocess}
\label{sec:mps}

Sempre al fine di preparare la discussione delle \acl{CTBN} si prosegue presentando alcuni concetti propedeutici relativi ai \mprocess{}, una categoria di processi stocastici \emph{memoryless} (con assenza di memoria).

\begin{definizione}[Proprietà di Markov]
\label{defn:markov-assumption}
Secondo la proprietà di Markov gli stati futuri di un processo stocastico sono indipendenti dagli stati passati, avendo evidenza sullo stato presente di tale processo.

Formalmente, se $\setel{X}(t),\:t > 0$ è un processo stocastico che gode di tale proprietà, allora $\forall\:h > 0$ vale la seguente equazione:
\begin{equation}
\label{eq:markov-assumption}
\set{P}(\setel{X}(t + h)\,|\,\setel{X}(s)=x(s),\,s\leq t)=\set{P}(\setel{X}(t + h)\,|\,\setel{X}(t)=x(t))
\end{equation}
Con la caratterizzazione \emph{assunzione di Markov} si è soliti descrivere i modelli che rispettano tale proprietà.
\end{definizione}

Di conseguenza possiamo osservare che la \acl{CPD} degli stati futuri di un processo stocastico che gode di tale proprietà dipende solo ed esclusivamente dallo stato attuale del processo, data la conoscenza dei suoi stati passati.

In altri termini ciò indica che lo stato futuro di una variabile casuale è \emph{condizionalmente indipendente} (si veda \ref{defn:ic}) dalla sequenza dei suoi stati passati, avendo evidenza sul suo stato presente.

Dalla proprietà di Markov deriva la definizione di \mprocess{}.

\begin{definizione}[\mprocess{}]
Si definisce come Markov process un processo stocastico che gode della proprietà di Markov.
\end{definizione}

Esistono due tipi di processi di Markov: omogenei e non. Si procede quindi dandone le definizioni.

\begin{definizione}[\homm\mprocess]
\label{defn:homogeneus-markov-process}
Un processo di Markov è detto \emph{omogeneo} qualora esso $\forall\:t,\,h > 0$ non dipenda dal tempo $t$, ovvero:
\begin{equation}
\label{eq:homogeneus-markov-process}
\set{P}(\setel{X}(t + h)\,|\,\setel{X}(t)=x(t))=\set{P}(\setel{X}(h)\,|\,\setel{X}(0)=x(0))
\end{equation}
\end{definizione}
Data quindi una variabile casuale $\setel{X}$ e l'insieme delle sue istanziazioni $val(\setel{X})=\{ \vectel{x_1}, \dotsc, \vectel{x_J} \}$, $\setel{X}(t)$ è un processo di Markov \emph{omogeneo, a tempo continuo e stati finiti} se e solo se la sua dinamica è definibile in termini di:
\begin{itemize}
    \item una distribuzione di probabilità iniziale $\priorsign{\set{X}}$ su $val(\setel{X})$
    \item una matrice di intensità $\imsign{\setel{X}}$.
\end{itemize}

\begin{definizione}[Matrice di intensità]
\label{defn:im}
Una matrice di intensità, o \acf{IM}, rappresenta un \emph{modello di transizione Markoviano}
\[
\imsign{\setel{X}}= \begin{bmatrix}
                        -q_{x_1}    & q_{x_1x_2}& \cdots & q_{x_1x_k}   \\[0.5em]
                        q_{x_2x_1}  & -q_{x_2}  & \cdots & q_{x_2x_k}   \\[0.5em]
                        \vdots      & \vdots    & \ddots & \vdots       \\[0.5em]
                        q_{x_kx_1}  & q_{x_kx_2}& \cdots & -q_{x_k}
                    \end{bmatrix}
\]
il cui scopo è rappresentare e descrivere il comportamento transiente di $\setel{X}$, un processo di Markov omogeneo.
\end{definizione}

Affinché $\imsign{\setel{X}}$ sia una \acl{IM} valida, ogni sua riga deve sommare a $0$. Ciò è possibile solo se:
\[
q_{x_i}=\sum_{i \neq j}q_{x_{ij}}\quad\text{con}\quad q_{x_i}\,,\,q_{x_{ij}}>0
\]

Data quindi una matrice di intensità $\imsign{\setel{X}}$ essa descrive il comportamento transiente di $\setel{X}(t)$. Se $\setel{X}(0)=x_i$ allora il processo di Markov omogeneo (e indicizzato dal tempo $t$) $\setel{X}(t)$ rimarrà nello stato $x_i$ una quantità di tempo \emph{esponenzialmente distribuita} rispetto al parametro $q_{x_i}$. Di conseguenza la \emph{funzione di densità} $f$ e la corrispondente \emph{funzione di ripartizione}\footnote{Nel calcolo delle probabilità la funzione di ripartizione di una variabile casuale $X$ a valori reali, anche nota come funzione di distribuzione cumulativa, è la funzione che associa a ciascun valore $x$ la probabilità che $X$ assuma valori minori o uguali ad $x$.} $F$ sono:
\begin{equation}
\label{eq:im-distrib}
\begin{split}
f(t) &= q_{x_i}\,exp(-q_{x_i}\,t)\,,\quad t>0 \\
F(t) &= 1-exp(-q_{x_i}\,t)\,,\quad t\geq0
\end{split} 
\end{equation}

Mentre gli elementi sulla diagonale, $q_{x_i}$, codificano la \emph{probabilità istantanea} che $\setel{X}$ abbandoni lo stato $x_i$, gli elementi non sulla diagonale\footnote{Elementi della matrice per i cui indici $i$ e $j$ risulta vero che $i \neq j$.}, $x_{ij}$, esprimono l'\emph{intensità di transizione} dallo stato $x_i$ allo stato $x_j$. 

Possiamo quindi calcolare:
\begin{itemize}
    \item il \emph{tempo atteso di una transizione uscente} dallo stato $x_i$ \[1/q_{x_i}\]
    \item la \emph{probabilità istantanea di transizione} dallo stato $x_i$ allo stato $x_j$ \[\theta_{x_{ij}}=q_{x_{ij}}/q_{x_i}\text{.}\]
\end{itemize}
Quindi una matrice di intensità induce una distribuzione di probabilità locale fattorizzata in due parti:
\begin{itemize}
    \item $q_{x_i}$, una \emph{distribuzione esponenziale}, che esprime \emph{quando} avvengono le transizioni
    \item $\theta_{x_{ij}}$, una \emph{distribuzione multinomiale}, che esprime \emph{verso} quale stato avvengono le transizioni.
\end{itemize}

Si osservi infine come la matrice $\imsign{\setel{X}}$ fa in modo che $\setel{X}$ soddisfi la proprietà di Markov poiché il comportamento futuro di $\setel{X}$ è definito solamente in base al suo stato attuale (\eqref[vale l'equazione]{eq:homogeneus-markov-process}).

\begin{definizione}[\conm\mprocess]
\label{defn:conditional-markov-process}
Un processo di Markov le cui intensità di transizione variano nel tempo non in funzione del tempo ma in funzione dei valori assunti ad ogni determinato istante $t$ da un insieme di altre variabili, che evolvono anch'esse come dei processi di Markov, è detto essere un \conm\mprocess{} (o Inhomogeneous \mprocess{}).

Assumendo quindi che una variabile casuale $\setel{X}$ evolva come un processo di Markov $\setel{X}(t)$ e che la sua dinamica sia condizionata da un insieme di altre variabili casuali $\set{U}$, anch'esse dei processi di Markov, possiamo definire per tale variabile casuale una \acf{CIM} $\cimsign{\setel{X}}{\set{U}}$.

Se specifichiamo una distribuzione di probabilità iniziale su $\setel{X}$ abbiamo così definito un \mprocess{} il cui comportamento dipende dalle istanziazioni dei valori di $\set{U}$.
\end{definizione}

\begin{definizione}[Matrice di intensità condizionale]
\label{defn:cim}
Dato un insieme di processi di Markov $\set{U}$, una matrice di intensità condizionale $\cimsign{\setel{X}}{\set{U}}$ è costituita da un insieme di matrici di intensità $\cimsign{\setel{X}}{\setel{u_i}}$, una per ogni diversa istanziazione $\setel{u_i}$ di $\set{U}$.
\[
\cimsign{\setel{X}}{\set{U}}=\big\{\,\cimsign{\setel{X}}{\setel{u_1}}\,,\,\cimsign{\setel{X}}{\setel{u_2}}\,,\,\dotsc\,,\,\cimsign{\setel{X}}{\setel{u_n}}\,\big\}
\]
dove\\
\[
\cimsign{\setel{X}}{u_i}
    =  \begin{bmatrix}
        -q_{x_1}^{u_i}   & q_{x_1x_2}^{u_i} & \cdots & q_{x_1x_k}^{u_i}\\[0.5em]
        q_{x_2x_1}^{u_i} & -q_{x_2}^{u_i}   & \cdots & q_{x_2x_k}^{u_i}\\[0.5em]
        \vdots           & \vdots           & \ddots & \vdots          \\[0.5em]
        q_{x_kx_1}^{u_i} & q_{x_kx_2}^{u_i} & \cdots & -q_{x_k}^{u_i}
    \end{bmatrix}
\]
\end{definizione} 

%%%%%%%%%%%%%%%%%%%%%%%%%%%%%%%%%%%%%%%%%%%%%%%%%%%%%%%%%%%%%%%%%%%%%%%%%%%%%%%%
\section{Definizioni preliminari}
\label{sec:Definizioni preliminari}
%%%%%%%%%%%%%%%%%%%%%%%%%%%%%%%%%%%%%%%%%%%%%%%%%%%%%%%%%%%%%%%%%%%%%%%%%%%%%%%%
Nelle precedenti sezioni sono stati illustrati i concetti che si pongono a fondamento delle \acl{CTBN}:
\begin{itemize}
    \item le \acl{BN} utili a comprendere la rappresentazione strutturata dello spazio degli stati delle \acs{CTBN}, l'utilizzo della nozione di indipendenza condizionale e le conseguenti tecniche di apprendimento e inferenza
    \item i \mprocess{}, omogenei e non, al fine di introdurre le modalità di rappresentazione (qualitativa e quantitativa) esplicita delle dinamiche temporali.
\end{itemize}

\`E ora possibile quindi presentare le \acl{CTBN} come una collezione di \ac{CTMP} non omogenei e con spazio degli stati discreto~\cite{Nodelman2007}.

\begin{definizione}[\acl{CTMP}]
Variabile casuale $\setel{X}(t)$, che gode della proprietà di Markov, indicizzata dal tempo $t\in[\,0,\,\infty)$.
\end{definizione}

Di seguito, invece, si riportano alcune definizioni che torneranno utili durante il prosieguo della discussione.

\begin{definizione}[\ACL{PV}]
    Una \acf{PV} $\set{X}$ è un insieme di \acl{CTMP} $\setel{X}(t)$.
\end{definizione}

\begin{definizione}[Traiettoria]
    Istanziazione di un insieme di valori per $\setel{X}(t)$ al variare di $t$.
\end{definizione}

\begin{definizione}[$J$-time-segment]
Partizionamento di un intervallo temporale $[\,0,\,T)$ in $J$ intervalli chiusi a sinistra:
\[
[\,0,\,t_1)\:;\:[\,t_1,\,t_2)\:;\:\dotsc\:;\:[\,t_{J-1},\,T)
\]
\end{definizione}

\begin{definizione}[$J$-evidence-stream]
Dato un $J$-time-segment composto da $J$ intervalli temporali e una \acl{PV} $\set{X}$ composta da $N$ variabili casuali, un $J$-evidence-stream è l'insieme delle istanziazioni comuni $\set{X}=\set{x}$ associate ad ogni $J$-time-segment per ogni sottoinsieme delle variabili casuali. \`E denotato con $(\set{X}^1=\set{x}^1\,,\,\set{X}^2=\set{x}^2\,,\,\dotsc\,,\,\set{X}^J=\set{x}^J)$, o più concisamente $(\set{x}^1\,,\,\set{x}^2\,,\,\dotsc\,,\,\set{x}^J)$.
\end{definizione}

Un $J$-evidence-stream $(\set{x}^1\,,\,\set{x}^2\,,\,\dotsc\,,\,\set{x}^J)$ è detto essere \emph{fully observed} (completamente osservato) se lo stato di tutte le variabili $\setel{X_n}\in\set{X}$ è conosciuto in tutto l'intervallo $[\,0,\,T)$. Viceversa, un $J$-evidence-stream è detto \emph{partially observed} (parzialmente osservato)~\cite{Stella2012}.

%%%%%%%%%%%%%%%%%%%%%%%%%%%%%%%%%%%%%%%%%%%%%%%%%%%%%%%%%%%%%%%%%%%%%%%%%%%%%%%%
\section{Rappresentazione}
\label{sec:ctbn-rappresentazione}
%%%%%%%%%%%%%%%%%%%%%%%%%%%%%%%%%%%%%%%%%%%%%%%%%%%%%%%%%%%%%%%%%%%%%%%%%%%%%%%%
Una \acl{CTBN} è un modello grafico in cui ogni nodo è associato con una variabile casuale i cui stati evolvono nel tempo continuo. Le dinamiche evolutive degli stati dei nodi sono governate e dipendono dal valore che gli stati dei nodi padre\footnote{Con il termine <<nodo padre>>, o \emph{parent node}, si intende un nodo il cui stato condiziona quello di un altro nodo del modello grafico.} assumono~\cite{Stella2012}. Quindi ogni nodo è un \conm\mprocess{} (\ref{defn:conditional-markov-process}) a tempo continuo e spazio degli stati discreto.

Una \acs{CTBN} è composta principalmente da due componenti:
\begin{itemize}
    \item una distribuzione di probabilità iniziale
    \item le dinamiche che regolano l'evoluzione nel tempo continuo della distribuzione di probabilità
\end{itemize}
Più formalmente si definisce:
\begin{definizione}[\acl{CTBN}]
\label{defn:ctbn}
Data una \ACL{PV} $\set{X}$, insieme di processi di Markov $\setel{X_1}\,,\,\setel{X_2}\,,\,\dotsc\,,\,\setel{X_N}$ a tempo continuo e con spazio degli stati finito $val(\setel{X_n})=\{\,\vectel{x_1}\,,\,\dotsc\,,\,\vectel{x_J}\,\}$ (dove $n=1\,,\,\dotsc\,,\,N$), una \acs{CTBN} $\conceptsym{N}$ su $\set{X}$ consiste di:
\begin{itemize}
    \item una distribuzione di probabilità iniziale $\priorsign{\set{X}}$ specificata come una \bn{} $\conceptsym{B}$ su $\set{X}$
    \item un modello di transizione continuo, specificato da:
    \begin{itemize}
        \item un grafo $\conceptsym{G}$, orientato e non necessariamente aciclico, composto dai nodi $\setel{X_1}\,,\,\setel{X_2}\,,\,\dotsc\,,\,\setel{X_N}$, ognuno dei quali possiede un insieme di genitori denotato da $pa(\setel{X_n})$
        \item una matrice di intensità condizionale $\cimsign{\setel{X_n}}{pa(\setel{X_n})}$ per ogni nodo $\setel{X_n} \in \set{X}$.
    \end{itemize}
\end{itemize}
\end{definizione}

Per ogni variabile causale $\setel{X_n} \in \set{X}$ di $\conceptsym{N}$ si ha quindi un insieme di modelli di probabilità locali: $\cimsign{\setel{X_n}}{pa(\setel{X_n})}$, la \acs{CIM} di $\setel{X_n}$, è infatti un insieme di modelli di transizione Markoviani (definiti tramite delle matrici di intensità \acs{IM}) la cui cardinalità è pari a quella dell'insieme delle diverse istanziazioni di $pa(\setel{X_n})$.

Si riscontra, quindi, quanto già affermato in precedenza (si veda \ref{sec:fondamenti-mp}), cioè che una \acs{CTBN}, fissato un ordinamento delle variabili da cui è costituita, esprime la sua dinamica evolutiva globale tramite un unico \homm\mprocess, costituito da un insieme di \conm\mprocess{} (un insieme di \acs{CIM} e relative distribuzioni di probabilità iniziali).

Si noti che, diversamente dalle \acl{BN}, nelle \acl{CTBN} gli archi fra i nodi rappresentano le relazioni temporali fra essi. Tali relazioni codificano le dinamiche evolutive dei nodi, ognuna delle quali è espressa condizionatamente alla dinamica evolutiva degli stati dei suoi nodi genitori, tramite le matrici di intensità condizionale. Per tale motivo è possibile che la componente $\conceptsym{G}$ del modello di transizione continuo contenga dei cicli. Tra l'altro, come vedremo nel prosieguo, la mancanza di tale vincolo di aciclicità porta a notevoli vantaggi computazionali relativamente all'apprendimento della struttura di una \acs{CTBN} dai dati.

%%%%%%%%%%%%%%%%%%%%%%%%%%%%%%%%%%%%%%%%%%%%%%%%%%%%%%%%%%%%%%%%%%%%%%%%%%%%%%%%
\section{Apprendimento}
\label{sec:ctbn-apprendimento}
%%%%%%%%%%%%%%%%%%%%%%%%%%%%%%%%%%%%%%%%%%%%%%%%%%%%%%%%%%%%%%%%%%%%%%%%%%%%%%%%
In questa sezione si affronta il processo di apprendimento dei parametri delle \acl{CTBN} da \emph{dati completi} mentre quello relativo a dati non completi viene tralasciato poiché non è argomento di questo lavoro di tesi.

Per \emph{dati completi} si intende un insieme $\conceptsym{D}=\{\,\delta_1\,,\,\dotsc\,,\,\delta_h\,\}$ composto da una o più traiettorie di tutte le variabili casuali di una \acs{CTBN}, dove ogni $\delta_i$ è un insieme completo (indicizzato dal tempo) di transizioni fra stati: per ogni momento temporale di ogni traiettoria conosciamo l'istanziazione di tutte le variabili casuali in questione.

Poiché le \acs{CTBN} sono un modello esponenziale la probabilità di un dataset può essere espressa in termini di \emph{sufficient statistics} aggregate, argomento che viene introdotto di seguito. Si mostra inoltre come la probabilità di una \acs{CTBN}, così come accade per le \acs{BN}, è un aggregato di probabilità locali.

\subsection{Likelihood di una singola transazione}
\label{sec:ctbn-data-likelihood-1}
Si consideri un singolo \homm\mprocess{} $\setel{X}(t)$.
Poiché tutte le transizioni sono osservabili, la \emph{likelihood}\footnote{Funzione di verosimiglianza di un evento, più debole della funzione di probabilità. Si applica solitamente ai parametri di un modello statistico. La likelihood di un insieme di valori per i parametri, dato un insieme di dati, è uguale alla probabilità dei dati, dati tali valori per i parametri.} del dataset $\conceptsym{D}$ può essere decomposta come un prodotto delle likelihood individuali di ogni singola transizione $d$.

Data quindi una tripla $d=\langle \,x_{d}\,,\,t_d\,,\,x_{d^{\prime}}\,\rangle\in\conceptsym{D}$ che esprime una transizione di $\setel{X}(t)$ da $x_d$ a $x_{d^{\prime}}$ dopo essere rimasto $t_d$ tempo in $x_d$, possiamo scrivere la likelihood di questa singola transizione $d$ in funzione dei parametri (\ref{sec:ctbn-params}):
\begin{equation}
\label{eq:ctbn-trans-hmm-likelihood}
\begin{split} 
L_{\setel{X}}(q\,,\,\theta:d) &= L_{\setel{X}}(q:d)\,L_{\setel{X}}(\theta:d) \\
&= q_{x_d}\,exp(-q_{x_d}\,t_d)\,(\theta_{x_{d}x_{d^{\prime}}})\text{.}
% &= q_{x_d}\,exp(-q_{x_d}\,t_d)\,\Big(\frac{q_{x_{d}x_{d^{\prime}}}}{q_{x_d}}\Big) \\
% &= exp(-q_{x_d}\,t_d)\,q_{x_{d}x_{d^{\prime}}}
\end{split}
\end{equation}

Si noti che \eqref[l'equazione]{eq:ctbn-trans-hmm-likelihood} è ricavata moltiplicando la \emph{p.d.f} di $\setel{X}(t)$ (\eqref[vedasi]{eq:im-distrib}) per la \emph{probabilità istantanea di transizione} (vedasi \ref{defn:im}).

Poiché moltiplicando le singole probabilità di transizione possiamo calcolare la probabilità dell'intero dataset $\conceptsym{D}$ possiamo quindi sintetizzare $\conceptsym{D}$ aggregando le \emph{sufficient statistics}.

\subsection{Sufficient Statistics}
\label{sec:ctbn-sufficient-stats}
Le \emph{sufficient statistics} per un singolo \homm\mprocess{} $\setel{X}(t)$ riassumono la sua dinamica evolutiva con:
\begin{itemize}
    \item $\tstat{x}$, la quantità di tempo trascorsa nello stato $x$
    \item $\mstat{x,x^\prime}$, il numero di transizioni dallo stato $x$ allo stato $x^\prime$.
\end{itemize}

Il numero totale di transizioni uscenti da uno stato $x$ è
\[
\mstat{x}=\sum_{x^\prime}\mstat{x,x^\prime}\text{.}
\]

Perciò possiamo ora calcolare la likelihood del dataset $\conceptsym{D}$ rispetto al singolo $\setel{X}(t)$.
\begin{equation}
\label{eq:ctbn-data-hmp-likelihood} 
\begin{split}
L_{\setel{X}}(q\,,\,\theta\,:\,\conceptsym{D}) &= \Bigg(\prod_{d\in\conceptsym{D}}L_{\setel{X}}(q:d) \Bigg)\Bigg(\prod_{d\in\conceptsym{D}}L_{\setel{X}}(\theta:d)\Bigg) \\
&= \Bigg(\prod_{x}q_x^{\mstat{x}}\,exp\Big(-q_x\,\tstat{x}\Big)\Bigg)\Bigg(\prod_{x}\prod_{x\neq x^{\prime}}\theta_{xx^{\prime}}^{\mstat{x,x^{\prime}}}\Bigg)\text{.}
\end{split}
\end{equation}

Si supponga ora di traslare questo concetto a una \acl{CTBN} $\conceptsym{N}$ con $N$ nodi: per ogni $\setel{X_i}$, con $i=1\,,\,\dotsc\,N$ è necessario considerare tutte le transazioni contestualmente all'istanziazione dell'insieme $\set{U}$ dei suoi nodi genitori. Poiché, nel caso di \emph{dati completi}, si conosce sempre l'istanziazione di $\set{U}$, allora, per ogni preciso momento nel tempo $t$, si conosce quale matrice di intensità (omogenea) $\cimsign{\setel{X_i}}{\setel{u}}$ governi la dinamica di $\setel{X_i}$.

Perciò la probabilità dei dati $\conceptsym{D}$ rispetto a $\conceptsym{N}$ è il prodotto delle likelihood di ogni variabile $\setel{X_n}$:
\begin{equation}
\begin{split}
L_{\conceptsym{N}}(q\,,\,\theta\,:\conceptsym{D}) &= \prod_{\setel{X_i}\in\set{X}}L_{\setel{X_i}}(q_{\setel{X_i}\,|\,\setel{u}}\,,\,\theta_{\setel{X_i}\,|\,\setel{u}}:\conceptsym{D}) \\
&= \prod_{\setel{X_i}\in\set{X}}L_{\setel{X_i}}(q_{\setel{X_i}\,|\,\setel{u}}:\conceptsym{D})L_{\setel{X_i}}(\theta_{\setel{X_i}\,|\,\setel{u}}:\conceptsym{D})
\end{split}
\end{equation}
Il termine $L_{\setel{X_i}}(\theta_{\setel{X_i}\,|\,\setel{u}}:\conceptsym{D})$ esprime la likelihood di una specifica sequenza di transizioni. Si osservi, di seguito, come il tempo che intercorre fra le transizioni sia trascurato poiché esse dipendono esclusivamente dal valore di nodi genitori. Usando le \emph{sufficient statistics} possiamo perciò scrivere:
\[
L_{\setel{X_i}}(\theta:\conceptsym{D}) = \prod_{u} \prod_{x} \prod_{x\neq x^{\prime}}\theta_{xx^{\prime}\,|\,u}^{\mstat[x]{x^{\prime}\,|\,u}}
\]
Tale funzione esprime la likelihood di tutte le sequenze di transizione.

Ora si affronta, invece, il calcolo di $L_{\setel{X_i}}(q_{\setel{X_i}\,|\,\setel{u}}:\conceptsym{D})$. Innanzitutto si consideri il caso in cui il tempo trascorso in uno determinato stato $x$ da $X_i$ termini non a causa di un suo cambiamento di valore (transizione) ma bensì a causa di una transizione di uno o più nodi appartenenti all'insieme dei suoi nodi genitori (\ie{} una nuova istanziazione per l'insieme dei genitori). Ebbene, è quindi necessario includere la probabilità che $X_i$ rimanga in $x$ (mentre i nodi genitori non effettuano alcuna transizione di stato) una quantità di tempo almeno pari a $t$. Tale quantità si ricava dalla funzione di distribuzione cumulativa di una distribuzione esponenziale \eqref[(si veda l'equazione]{eq:im-distrib}):
\[
1-F(t)=exp(-q_{x\,|\,u}\,t)\text{.}
\]

Perciò, si ottiene:
\begin{itemize}
    \item $\tstat{x\,|\,u}$, la quantità di tempo trascorsa da $\setel{X}(t)$ nello stato $x$ quando $\set{U}=u$
    \item $\mstat[x]{x^\prime\,|\,u}$, il numero di transizioni dallo stato $x$ allo stato $x^\prime$ di $\setel{X}(t)$ quando $\set{U}=u$.
\end{itemize}


% TODO: come calcolare una CIM dalle SS -> 5.1

\subsection{Stima dei parametri}
\label{sec:ctbn-params}
Introdotte le \emph{sufficient statistics} si approccia ora il problema dell'apprendimento dei parametri di una \acl{CTBN} di cui conosciamo la struttura da un insieme di dati completi.
I parametri, infatti, sono funzione delle \emph{sufficient statistics}.
% TODO: 5.1 da Nodelman2007

\subsection{Algoritmo di apprendimento}
...

%%%%%%%%%%%%%%%%%%%%%%%%%%%%%%%%%%%%%%%%%%%%%%%%%%%%%%%%%%%%%%%%%%%%%%%%%%%%%%%%
\section{Inferenza}
\label{sec:ctbn-inferenza}
%%%%%%%%%%%%%%%%%%%%%%%%%%%%%%%%%%%%%%%%%%%%%%%%%%%%%%%%%%%%%%%%%%%%%%%%%%%%%%%%
...

% TODO: sezione 8 da Nodelman2007 ?

%%%%%%%%%%%%%%%%%%%%%%%%%%%%%%%%%%%%%%%%%%%%%%%%%%%%%%%%%%%%%%%%%%%%%%%%%%%%%%%%
\section{Apprendimento strutturale}
\label{sec:ctbn-apprendimento-strutturale}
%%%%%%%%%%%%%%%%%%%%%%%%%%%%%%%%%%%%%%%%%%%%%%%%%%%%%%%%%%%%%%%%%%%%%%%%%%%%%%%%
...
% TODO: 5.2 da Nodelman2007, paper Nodelman2002

\subsection{Score} % funzione di valutazione
\label{sec:ctbn-score}

\subsection{Ricerca della struttura}
\label{sec:ctbn-graph-search}
% TODO: bla bla bla

\subsubsection{Hill Climbing}
...

%%%%%%%%%%%%%%%%%%%%%%%%%%%%%%%%%%%%%%%%%%%%%%%%%%%%%%%%%%%%%%%%%%%%%%%%%%%%%%%%

% TODO: inserire riferimento/i per la sezione apprendimento e inferenza delle BN (Pearl, Heckermann 1995, Yedidia2000 per generalized belief propagation)
% TODO: inserire riferimento/i per la sezione sui Markov process
% TODO: inserire immagini esemplificative delle ctbn
% TODO: controllare da 2.2 di Nodelman2007 la sezione sulle BN
% TODO: inserire esempi relativi alle im/cim?
% TODO: prima o dopo queste 2 def parlare della point evidence (vedi paper stella private)

%%%%%%%%%%%%%%%%%%%%%%%%%%%%%%%%%%%%%%%%

% Le CTBN rappresentano esplicitamente il tempo: non è necessario
% scegliere la granularità temporale a priori (contro DBNs)

% Le variabili casuali devono rappresentare aspetti distinti del mondo
% perciò si assume che esse non possano effettuare transizioni nello
% stesso istante


% TODO: dove servono, se servono?

