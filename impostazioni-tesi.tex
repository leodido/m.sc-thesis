%*********************************************************************************
% impostazioni-tesi.tex
% file che contiene le impostazioni della tesi
%*********************************************************************************

%*********************************************************************************
% Comandi personali
%*******************************************************
\newcommand{\myName}{Leonardo Di Donato}% autore
\newcommand{\myTitle}{Continuos Time Bayesian Networks Classifier}% titolo
\newcommand{\myDegree}{Tesi di laurea magistrale}% tipo di tesi
\newcommand{\myUni}{Universit� degli Studi di Milano--Bicocca}% universit�
\newcommand{\myFaculty}{Facolt� di Scienze Matematiche, Fisiche e Naturali}% facolt�
\newcommand{\myDepartment}{Dipartimento di Informatica, Sistemistica e Comunicazione}% dipartimento
\newcommand{\myProf}{Prof.~F.~Antoni~Stella}% relatore
\newcommand{\myOtherProf}{Dott.~Daniele~Codecasa}% correlatore
\newcommand{\myLocation}{Milano}% dove
\newcommand{\myTime}{Marzo 2013}% quando

%*********************************************************************************
% Impostazioni di amsmath, amssymb, amsthm
%*********************************************************************************
% comandi per gli insiemi numerici (serve il pacchetto amssymb)
\newcommand{\numberset}{\mathbb} 
\newcommand{\N}{\numberset{N}} 
\newcommand{\R}{\numberset{R}} 

% un ambiente per i sistemi
\newenvironment{sistema}%
  {\left\lbrace\begin{array}{@{}l@{}}}%
  {\end{array}\right.}

% definizioni (serve il pacchetto amsthm)
\theoremstyle{definition} 
\newtheorem{definizione}{Definizione}

% teoremi, leggi e decreti (serve il pacchetto amsthm)
\theoremstyle{plain} 
\newtheorem{teorema}{Teorema}
\newtheorem{legge}{Legge}
\newtheorem{decreto}[legge]{Decreto}
\newtheorem{murphy}{Murphy}[section]

% comandi per le lettere greche
\renewcommand{\epsilon}{\varepsilon}
\renewcommand{\theta}{\vartheta}
\renewcommand{\rho}{\varrho}
\renewcommand{\phi}{\varphi}

%*********************************************************************************
% Impostazioni di biblatex
%*********************************************************************************
\defbibheading{bibliography}{%
\cleardoublepage
\manualmark
\phantomsection 
\addcontentsline{toc}{chapter}{\tocEntry{\bibname}}
\chapter*{\bibname\markboth{\spacedlowsmallcaps{\bibname}}
{\spacedlowsmallcaps{\bibname}}}}

%*********************************************************************************
% Impostazioni di listings
%*********************************************************************************
\lstset{language=[LaTeX]Tex,%C++,
    keywordstyle=\color{RoyalBlue},%\bfseries,
    basicstyle=\small\ttfamily,
    %identifierstyle=\color{NavyBlue},
    commentstyle=\color{Green}\ttfamily,
    stringstyle=\rmfamily,
    numbers=none,%left,%
    numberstyle=\scriptsize,%\tiny
    stepnumber=5,
    numbersep=8pt,
    showstringspaces=false,
    breaklines=true,
    frameround=ftff,
    frame=single
}

%*********************************************************************************
% Impostazioni di hyperref
%*********************************************************************************
\hypersetup{%
    hyperfootnotes=false,pdfpagelabels,
    %draft,	% = elimina tutti i link (utile per stampe in bianco e nero)
    colorlinks=true, linktocpage=true, pdfstartpage=1, pdfstartview=FitV,%
    % decommenta la riga seguente per avere link in nero (per esempio per la stampa in bianco e nero)
    %colorlinks=false, linktocpage=false, pdfborder={0 0 0}, pdfstartpage=1, pdfstartview=FitV,% 
    breaklinks=true, pdfpagemode=UseNone, pageanchor=true, pdfpagemode=UseOutlines,%
    plainpages=false, bookmarksnumbered, bookmarksopen=true, bookmarksopenlevel=1,%
    hypertexnames=true, pdfhighlight=/O,%nesting=true,%frenchlinks,%
    urlcolor=webbrown, linkcolor=RoyalBlue, citecolor=webgreen, %pagecolor=RoyalBlue,%
    %urlcolor=Black, linkcolor=Black, citecolor=Black, %pagecolor=Black,%
    pdftitle={\myTitle},%
    pdfauthor={\textcopyright\ \myName, \myUni, \myFaculty},%
    pdfsubject={},%
    pdfkeywords={},%
    pdfcreator={pdfLaTeX},%
    pdfproducer={LaTeX with hyperref and classicthesis}%
}

%*********************************************************************************
% Impostazioni di graphicx
%*********************************************************************************
\graphicspath{{immagini/}} % cartella dove sono riposte le immagini

%*********************************************************************************
% Margini ottimizzati per l'A4
%*********************************************************************************
\areaset[current]{336pt}{750pt}
\setlength{\marginparwidth}{7em}
\setlength{\marginparsep}{2em}%

%*********************************************************************************
% Acronimi
%*********************************************************************************
%\PassOptionsToPackage{printonlyused,smaller}{acronym}
%  \usepackage{acronym} % nice macros for handling all acronyms in the thesis
%\renewcommand*{\acsfont}[1]{\textssc{#1}} % for MinionPro
%\renewcommand{\bflabel}[1]{{#1}\hfill} % fix the list of acronyms

%*********************************************************************************
% Altro
%*********************************************************************************

% ...
\newcommand{\omissis}{[\dots\negthinspace]}

% eccezioni all'algoritmo di sillabazione
\hyphenation{Fortran ma-cro-istru-zio-ne nitro-idrossil-amminico}