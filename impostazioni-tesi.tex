% !TEX encoding = UTF-8
% !TEX TS-program = pdflatex
% !TEX root = tesi.tex
% !TEX spellcheck = it-IT

% impostazioni-tesi.tex
% file che contiene le impostazioni della tesi
%*********************************************************************************

%*********************************************************************************
% Comandi personali
%*******************************************************
\newcommand{\myName}{Leonardo~Di~Donato}% autore
\newcommand{\myTitle}{Continuos Time Bayesian Networks Classifiers}% titolo
\newcommand{\myDegree}{Tesi di Laurea}% tipo di tesi
\newcommand{\myUni}{Universit� degli Studi di Milano--Bicocca}% universit�
\newcommand{\myFaculty}{Facolt� di Scienze Matematiche, Fisiche e Naturali}% facolt�
\newcommand{\myDepartment}{Dipartimento di Informatica, Sistemistica e Comunicazione}% dipartimento
\newcommand{\myProf}{Prof.~F.~Antonio~Stella}% relatore
\newcommand{\myOtherProf}{Dott.~Daniele~Codecasa}% correlatore
\newcommand{\myLocation}{Milano}% dove
\newcommand{\myTime}{Luglio 2013}% quando
\newcommand{\mySubject}{Continuos time Bayesian Networks}
\newcommand{\myKeywords}{Continuos time Bayesian Network, CTBN, Continuos time Bayesian Newtork Classifier, CTBNC}

%*********************************************************************************
% Impostazioni di amsmath, amssymb, amsthm
%*********************************************************************************
% comandi per gli insiemi numerici (serve il pacchetto amssymb)
\newcommand{\numberset}{\mathbb} 
\newcommand{\N}{\numberset{N}} 
\newcommand{\R}{\numberset{R}} 

% comandi per gli insiemi
\newcommand{\set}[1]{\mathbf{#1}}
\newcommand{\setel}[1]{\mathnormal{#1}}  % o \mathrm o \mathit

% comandi per i vettori
\newcommand{\vect}[1]{\set{#1}}
\newcommand{\vectel}[1]{\mathrm{#1}}

% comandi per i simboli matematici che denotano schemi, categorie, concetti
\newcommand{\conceptsym}[1]{\mathcal{#1}}

% un ambiente per i sistemi
\newenvironment{sistema}%
  {\left\lbrace\begin{array}{@{}l@{}}}%
  {\end{array}\right.}

% definizioni (serve il pacchetto amsthm)
\theoremstyle{definition} 
\newtheorem{definizione}{Definizione}[section]
\newtheorem{osservazione}{Osservazione}[section]

% note
\theoremstyle{remark}
\newtheorem{nota}{Nota}[section]

% teoremi (serve il pacchetto amsthm)
\theoremstyle{plain} 
\newtheorem{teorema}{Teorema}
\newtheorem{cor}[teorema]{Corollario} 
\newtheorem{lem}[teorema]{Lemma}
\newtheorem{prop}[teorema]{Proposizione}
\newtheorem{oss}[teorema]{Osservazione}

% leggi (serve il pacchetto amsthm)
\newtheorem{legge}{Legge}
\newtheorem{decreto}[legge]{Decreto}

% comandi per le lettere greche
% ...

%*********************************************************************************
% Impostazioni di acronym
%*********************************************************************************
\newcommand{\acroname}{Acronimi}
% \renewcommand*{\acsfont}[1]{\textssc{#1}}                 % for MinionPro
\renewcommand*{\acsfont}[1]{\textsmaller{#1}}               % customize font for long version acronyms [works only if footnote not activate]
\renewcommand*{\acffont}[1]{#1}                             % idem, but for short version of acronyms
\renewcommand{\bflabel}[1]{{#1}\hfill}                      % fix the list of acronyms                                
\makeatletter                                               % macro that tweeks acronym package to rendere lowercase or not
\newif\if@in@acrolist
\AtBeginEnvironment{acronym}{\@in@acrolisttrue}
\newrobustcmd{\ul}[2]{\if@in@acrolist#1\else#2\fi}          % \ul{C}{c}iao defines an uppercase and lowercase variant of the same acronym
\newcommand{\ACF}[1]{{\@in@acrolisttrue\acf{#1}}}           % \ACF{<acronym>} force the defined capitalized variants of acronym letters/words
\makeatother

%*********************************************************************************
% Impostazioni di biblatex
%*********************************************************************************
\defbibheading{bibliography}{%
\cleardoublepage
\manualmark
\phantomsection 
\addcontentsline{toc}{chapter}{\tocEntry{\bibname}}
\chapter*{\bibname\markboth{\spacedlowsmallcaps{\bibname}}
{\spacedlowsmallcaps{\bibname}}}}

%*********************************************************************************
% Impostazioni di listings
%*********************************************************************************
\lstset{language=[LaTeX]Tex,%C++,
    keywordstyle=\color{RoyalBlue},%\bfseries,
    basicstyle=\small\ttfamily,
    %identifierstyle=\color{NavyBlue},
    commentstyle=\color{Green}\ttfamily,
    stringstyle=\rmfamily,
    numbers=none,%left,%
    numberstyle=\scriptsize,%\tiny
    stepnumber=5,
    numbersep=8pt,
    showstringspaces=false,
    breaklines=true,
    frameround=ftff,
    frame=single
}

%*********************************************************************************
% Impostazioni di hyperref
%*********************************************************************************
\hypersetup{%
    % hyperfootnotes=false,pdfpagelabels,
    %draft,	% = elimina tutti i link (utile per stampe in bianco e nero)
    colorlinks=true, linktocpage=true, pdfstartpage=1, pdfstartview=FitV,%
    % decommenta la riga seguente per avere link in nero (per esempio per la stampa in bianco e nero)
    % colorlinks=false, linktocpage=false, pdfborder={0 0 0}, pdfstartpage=1, pdfstartview=FitV,% 
    breaklinks=true, pdfpagemode=UseNone, pageanchor=true, pdfpagemode=UseOutlines,%
    plainpages=false, bookmarksnumbered, bookmarksopen=true, bookmarksopenlevel=1,%
    hypertexnames=true, pdfhighlight=/O,%nesting=true,%frenchlinks,%
    urlcolor=webbrown, linkcolor=RoyalBlue, citecolor=webgreen, %pagecolor=RoyalBlue,%
    %urlcolor=Black, linkcolor=Black, citecolor=Black, %pagecolor=Black,%
    pdftitle={\myTitle},%
    pdfauthor={\textcopyright\ \myName, \myUni, \myFaculty},%
    pdfsubject={\mySubject},%
    pdfkeywords={\myKeywords},%
    pdfcreator={pdfLaTeX},%
    pdfproducer={LaTeX with hyperref, classicthesis and arsclassica}%
}

%*********************************************************************************
% Impostazioni di graphicx
%*********************************************************************************
\graphicspath{{immagini/}} % cartella dove sono riposte le immagini

%*********************************************************************************
% Margini ottimizzati per l'A4
%*********************************************************************************
\areaset[current]{336pt}{750pt}
\setlength{\marginparwidth}{7em}
\setlength{\marginparsep}{2em}%

%*********************************************************************************
% Utility
%*********************************************************************************
% make first letter uppercase
\makeatletter
\def\upcase{\expandafter\makeupcase}
\def\makeupcase#1{\uppercase{#1}}
\makeatother

% make first letter lowercase
\makeatletter
\def\lwcase{\expandafter\makelwcase}
\def\makelwcase#1{\lowercase{#1}}
\makeatother

%*********************************************************************************
% Equation references
%*********************************************************************************
\usepackage{etoolbox}
\let\originaleqref\eqref
\renewcommand\eqref[2][]{%
  \ifstrempty{#1}{%
    \ref{#2}%
  }{%
    #1~\ref{#2}%
  }%
}

%*********************************************************************************
% Altro
%*********************************************************************************
% ...
\newcommand{\omissis}{\dots\negthinspace}

% 
\newcommand{\ie}{i.\,e.}
\newcommand{\Ie}{I.\,e.}
\newcommand{\eg}{e.\,g.}
\newcommand{\Eg}{E.\,g.} 

% parole comuni
\newcommand{\bn}{Bayesian Network}
\newcommand{\ctbn}{Continuos time \bn{}}
\newcommand{\ctbnc}{Continuos time \bn{} Classifier}
\newcommand{\im}{Intensity Matrix}
\newcommand{\cim}{Conditional \im{}}
\newcommand{\mprocess}{Markov Process}
\newcommand{\homm}{Homogeneus }
\newcommand{\conm}{Conditional }

% elementi matematici comuni
\newcommand{\imsign}[1]{\set{Q}_{#1}}
\newcommand{\cimsign}[2]{\imsign{\,#1|#2}}
\newcommand{\priorsign}[1]{\set{P}^0_{#1}}

%*********************************************************************************
% Eccezioni all'algoritmo di sillabazione
%*********************************************************************************
\hyphenation{Fortran ma-cro-istru-zio-ne nitro-idrossil-amminico}


