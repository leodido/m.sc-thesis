% !TEX encoding = UTF-8
% !TEX TS-program = pdflatex
% !TEX root = ../tesi.tex
% !TEX spellcheck = it-IT

%************************************************
\chapter{Ipsum}
\label{cap:ipsum}
%************************************************


Lorem ipsum dolor sit amet, consectetuer adipiscing elit. Nam dui ligula, fringilla a, euismod sodales, sollicitudin vel, wisi. Morbi auctor lorem non justo. Nam lacus libero, pretium at, lobortis vitae, ultricies et, tellus.
\begin{description}
\item[Lorem ipsum dolor] sit amet, consectetuer adipiscing elit. Ut purus elit, vestibulum ut, placerat ac $\lim_{n \to \infty}\sum_{k=1}^n \frac{1}{k^2}= \frac{\pi^2}{6}$.
\item[Mauris ut leo.]
Cras viverra metus rhoncus sem. Nulla et lectus vestibulum urna fringilla ultrices. Phasellus eu tellus sit amet tortor gravida placerat.
\[
\lim_{n \to \infty}\sum_{k=1}^n \frac{1}{k^2}= \frac{\pi^2}{6}.
\]
\end{description}

Nulla malesuada porttitor diam. Donec felis erat, congue non, volutpat at, tincidunt tristique, libero. Vivamus viverra fermentum felis.
\begin{equation}
\label{eq:euler}
e^{i\pi}+1=0.
\end{equation}
Dalla formula~\eqref{eq:euler} 
si deduce che\dots






\section{Nozioni basilari}

\subsection{Insiemi numerici}

Donec nonummy pellentesque ante. Phasellus adipiscing semper elit.
\begin{equation}
x^2 \geq 0 \quad
\forall x \in \mathbb{R}.
\end{equation}


\subsection{Le matrici}

\lipsum[2]
\begin{equation}
A=
\begin{bmatrix}
x_{11} & x_{12} & \dots \\
x_{21} & x_{22} & \dots \\
\vdots & \vdots & \ddots
\end{bmatrix}
\end{equation}



\section{Formule fuori corpo}

Proin fermentum massa ac quam. Sed diam turpis, molestie vitae, placerat a, molestie nec, leo. Maecenas lacinia. Nam ipsum ligula, eleifend at, accumsan nec, suscipit a, ipsum. 


\subsection{Una formula spezzata con allineamento}

\lipsum[2]
\begin{equation} 
\begin{split} 
a &= b+c-d \\ 
  &= e-f \\ 
  &= g+h \\ 
  &= i. 
\end{split} 
\end{equation}

 
\subsection{Casi}

\lipsum[3]
\begin{equation}
f(n):=
\begin{cases} 
2n+1, & \text{con $n$ dispari,} \\ 
n/2,  & \text{con $n$ pari.} 
\end{cases} 
\end{equation}



\section{Enunciati e dimostrazioni}

Nunc eleifend consequat lorem. Sed lacinia nulla vitae enim. Pellentesque tincidunt purus vel magna. Integer non enim. Praesent euismod nunc eu purus.
\begin{definizione}[di Gauss] 
Un \emph{matematico} trova ovvio che
$\int_{-\infty}^{+\infty}
e^{-x^2}\,dx=\sqrt{\pi}$. 
\end{definizione} 
\begin{teorema} 
I matematici, se ce ne sono, sono molto rari.
\end{teorema} 

\lipsum[2]

\begin{teorema}[di Pitagora]
La somma dei quadrati costruiti sui cateti uguaglia il quadrato costruito sull'ipotenusa.
\end{teorema}
La dimostrazione viene lasciata per esercizio.

Donec bibendum quam in tellus. Nullam cursus pulvinar lectus. Donec et mi. Nam vulputate metus eu enim. Vestibulum pellentesque felis eu massa.
\begin{teorema}[Sorpresa]
Si ha che $\log(-1)^2=2\log(-1)$.
\end{teorema} 
\begin{proof} 
Si ha che $\log(1)^2 = 2\log(1)$.
Ma si ha anche che $\log(-1)^2=\log(1)=0$.
Quindi $2\log(-1)=0$, da cui la tesi.
\end{proof}
Viene un quadratino a fine dimostrazione.
\begin{legge}
\label{lex:capo}
Il capo ha ragione.
\end{legge}
\begin{decreto}[Aggiornamento alla legge~\ref{lex:capo}]
Il capo ha \emph{sempre} ragione.
\end{decreto}
\begin{legge}
Se il capo ha torto, vedere la 
legge~\ref{lex:capo}.
\end{legge}


Nam dui ligula, fringilla a, euismod sodales, sollicitudin vel, wisi. Morbi auctor lorem non justo. Nam lacus libero, pretium at, lobortis vitae, ultricies et, tellus.
