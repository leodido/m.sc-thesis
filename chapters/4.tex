% !TEX encoding = UTF-8
% !TEX TS-program = pdflatex
% !TEX root = ../arsclassica.tex
% !TEX spellcheck = it-IT

%************************************************
\chapter{Package R}
\label{cap:ctbnr}
%************************************************
\omissis{}

\section{R}
\omissis{}

\section{Analisi}
\omissis{}

\section{Package CTBN}
\omissis{}

\subsection{Gestione dei dati}
\omissis{}

\subsection{Apprendimento}
\omissis{}

\subsection{Inferenza}
\omissis{}

\subsection{Apprendimento strutturale}
\omissis{}

\section{Cross-validazione}
\omissis{}

\subsection{Metriche di valutazione}
\omissis{}

\subsection{Package xvalidation}
\omissis{}

% introdurre R? Rcpp?
% introdurre paradigma funzionale?
% architettura/componenti
% signature + documentazione funzioni principali?
% parlare dei task che supportano la parallelizzazione
% struttura package R
% componente cross-validation: spiegare metriche di valutazione utilizzate? si! ma in capitolo 4!
