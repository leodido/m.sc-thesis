% !TEX encoding = UTF-8
% !TEX TS-program = pdflatex
% !TEX root = ../arsclassica.tex
% !TEX spellcheck = it-IT

%************************************************
\chapter{Creazione di dataset relativi al traffico}
\label{cap:tsis-sensors}
%************************************************
Al fine di valutare le prestazioni degli algoritmi di apprendimento e classificazione delle \acl{CTBN} è emersa la necessità di generare dei dataset adeguati a tale scopo.

Come già specificato in precedenza, le \acs{CTBN} sono un modello stocastico dedito alla rappresentazione dell'evoluzione di sistemi dinamici, cioè di fenomeni cangianti nel tempo (non necessariamente o esclusivamente in base ad esso) rappresentabili come insiemi di traiettorie multi-variate. Si è quindi scelto di generare dei dataset che rappresentassero un tipico sistema cangiante: il traffico automobilistico su rete stradale.

Tali dataset, costituiti da un insieme di documenti rappresentanti la presenza (durante l'evolvere del tempo) di veicoli sui sensori di una rete stradale, sono stati generati con l'ausilio di un software commerciale, \acf{TSIS} (versione $\geq$ 6.2) e di una sua \acl{RTE} appositamente sviluppata al fine di monitorare e tracciare il passaggio dei veicoli. 

In questo capitolo si presentano sia i succitati strumenti utilizzati per la creazione di reti stradali e relativi modelli di simulazione, sia lo strumento creato per la creazione di dataset relativi al traffico.

Relativamente, invece, al processo pratico di creazione dei dataset si rimanda alla~\autoref{sec:create-dataset-howto}.

\section{TSIS}
\label{sec:tsis}
\acf{TSIS} è un \acl{IDE}\footnote{Un \acl{IDE}, comunemente chiamato anche \ACF{IDE}, è un insieme di programmi finalizzati a supportare il processo di sviluppo dei software. Generalmente, un \acs{IDE} è costituito da uno strumento per la creazione e modifica del codice sorgente, un compilatore o un interprete, strumenti per l'automazione dello sviluppo e la qualità del codice sorgente.}, distribuito commercialmente da McTrans\footnote{McTrans Moving Technology: \url{http://mctrans.ce.ufl.edu}} e supportato dalla \acf{FWHA}\footnote{Agenzia del Dipartimento dei Trasporti degli Stati Uniti d'America: \\ \url{www.fhwa.dot.gov}}, il cui scopo ultimo è permettere la simulazione e l'analisi di modelli di reti stradali.

\acs{TSIS} è costituito da insieme di strumenti dedicati alla creazione di reti stradali e relativi modelli di simulazione, all'esecuzione, e eventualmente alla visualizzazione, di tali modelli, così come all'interpretazione dei risultati ottenuti. Tale insieme di strumenti è reso accessibile tramite delle interfaccie grafiche\footnote{Un'interfaccia grafica, nota anche come \acf{GUI}, ...}.

L'architettura modulare con cui \acs{TSIS} è realizzato permette, in caso di necessità, di estendere tale ambiente creando degli ulteriori strumenti.

Di seguito si introducono brevemente i concetti relativi a \acs{TSIS} utilizzati nel prosieguo di questo lavoro di tesi.

Tuttavia si osservi che, poiché lo scopo di questo capitolo non consiste nel documentare \acs{TSIS}, la trattazione dei suoi dettagli tecnici (\eg{}, significato e lista dei tipi di dati rappresentabili) è omessa. A tale fine si rimanda invece alla documentazione ufficiale del software in questione.

\begin{definizione}[Progetto \acs{TSIS}]\label{defn:tsis-proj}
Un progetto \acs{TSIS} è un insieme di modelli di simulazione per una specifica rete stradale.
\end{definizione}

\begin{definizione}[Modello di simulazione \acs{TSIS}]\label{defn:tsis-sim-model}
Un modello di simulazione è costituito da un input (\eg{}, variazioni dei flussi di ingresso nella rete, variazioni delle percentuali di svolta dei veicoli nelle intersezioni) per la simulazione di una determinata rete stradale e tutti i dati generati dalla sua esecuzione (\ie{}, simulazione).
\end{definizione}
\begin{osservazione}
Un modello di simulazione può anche prevedere che la sua esecuzione sia eseguita più volte. Finché il seme dei numeri casuali non è modificato, esso è sempre considerato un singolo modello di simulazione.
\end{osservazione}

\begin{definizione}[Formato \acs{TRF}]\label{defn:trf-format}
\acs{TRF}\footnote{\acf{TRF}} è il formato dei file accettati dal simulatore di \acs{TSIS}. Esso codifica e rappresenta una rete stradale specificandone i vari componenti tramite l'utilizzo dei rispettivi \acs{RT}\footnote{\acf{RT}}, cioè delle righe di testo contenenti un codice identificativo numerico e i valori per i rispettivi campi accettati nell'ordine prestabilito. Allo stesso modo, tale formato incorpora il modello di simulazione della rete stradale che descrive.
\end{definizione}
\begin{osservazione}
Il formato \acs{TRF} è equivalente al formato \acs{TRAF}\footnote{\acf{TRAF}}.
\end{osservazione}

\begin{definizione}[Formato \acs{TNO}]\label{defn:tno-format}
\acs{TNO}\footnote{\acf{TNO}} è il formato nativo con cui vengono rappresentate in memoria le reti stradali create visivamente tramite l'interfaccia grafica \acs{TRAFED}.
\end{definizione}

\subsection{Caratteristiche}
% TODO: panoramica features

\subsection{Componenti}

In questa sezione si elencano gli strumenti che costituiscono l'ambiente di sviluppo \acs{TSIS}.

\begin{description}
\item[CORSIM] \hfill \\
\acs{CORSIM}\footnote{\acf{CORSIM}} costituisce il componente principale dell'insieme di strumenti denominato \acs{TSIS}. È un simulatore il cui obiettivo è permettere la creazione e l'esecuzione di modelli di simulazione \acs{TSIS}. È composto da due simulatori integrati che rappresentano l'intero sistema di traffico come funzione del tempo: \acs{NETSIM}\footnote{\acf{NETSIM}} e \acs{FREESIM}\footnote{\acf{FREESIM}}. Tali simulatori integrati rappresentano, rispettivamente, il traffico sulle strade urbane e non. La simulazione effettuata da tali strumenti è di tipo microscopico: essi modellano individualmente il comportamento di ogni singolo veicolo, prendendo in considerazione per ognuno di essi una serie di variabili, anche di tipo stocastico (\eg{}, tipologia di guidatore). Per tale motivo \acs{CORSIM} è dotato di molte possibili opzioni di configurazione e permette lo studio di modelli molto complessi e dettagliati.
\item[TRAFED] \hfill \\
\acs{TRAFED}\footnote{\acf{TRAFED}} è una \acs{GUI} il cui scopo è permettere la creazione e la modifica di reti stradali e di modelli di simulazione per \acs{CORSIM}.
\item[TShell] \hfill \\
\acs{TShell}\footnote{\acf{TShell}} è la \acs{GUI} di \acs{TSIS}. Funge da contenitore degli strumenti (preconfigurati, o creati dall'utente) di questo ambiente di sviluppo integrato e permette la gestione dei progetti \acs{TSIS}.
\item[TRAFVU] \hfill \\
\acs{TRAFVU}\footnote{\acf{TRAFVU}} è una \acs{GUI} finalizzata alla visualizzazione dei modelli di simulazione simulati con \acs{CORSIM}. Essa permette sia di visualizzare in modo animato l'evoluzione del traffico nella rete stradale con una qualsiasi granularità temporale, sia di visualizzare una serie di misure di interesse relative alla simulazione.
\item[TSIS Text Editor] \hfill \\
\acs{TSIS} Text Editor è uno strumento il cui scopo è facilitare la modifica manuale dei file \acs{TRF}. A tale scopo esso visualizza per ogni \acs{RT} che si intende modificare sia la sua descrizione sia l'insieme dei campi supportati.
\item[TSIS Script Tool] \hfill \\
\acs{TSIS} Script Tool è uno strumento per la creazione, la modifica e l'esecuzione di codice \acs{VBS}\footnote{\acf{VBS}}. Questo strumento fornisce un meccanismo utile ad automatizzare le funzionalità di simulazione di \acs{TSIS} (\eg{}, esecuzioni multiple dello stesso modello di simulazione variando il seme dei numeri casuali che governa la distribuzione di ingresso dei veicoli nella rete stradale).
\item[TSIS Translator] \hfill \\
\acs{TSIS} Translator è uno strumento utile alla conversione dei file dal formato \acs{TRF} al formato \acs{TNO} e viceversa. Tale operazione risulta utile al fine di rendere i file \acs{TRF} utilizzabili tramite lo strumento \acs{TRAFED} così come per rendere i file \acs{TNO} utilizzabili con \acs{CORSIM}.
\item[TSIS Output Processor] \hfill \\
\acs{TSIS} Output Processor è uno strumento finalizzato alla raccolta e l'aggregazione dei dati da \acs{CORSIM} durante l'esecuzione multipla di modelli di simulazione. La sua caratteristica principale consiste nella computazione automatica di un insieme di statistiche predefinite. Esso permette di scegliere sia le statistiche di interesse sia la granularità temporale della loro computazione.
\item[CORSIM RTE] \hfill \\
\acs{TSIS} fornisce un meccanismo finalizzato all'estensione delle sue funzionalità tramite la creazione, da parte dell'utente, di altri strumenti da integrare nell'ambiente di sviluppo. Queste estensioni, dette \acs{CORSIM} \acs{RTE}\footnote{\acf{RTE}}, possono interfacciarsi direttamente con \acs{CORSIM}, modificando o aumentando una parte della logica di simulazione, collezionando dati o monitorando eventi speciali.
\end{description}

\subsection{Estendere TSIS}
% panoramica sviluppo RTE
% TODO: descrizione delle varie API fornite da TSIS
% spostare ultima definizione CORSIM RTE qui? si dai ..
% in questa sezione presentaimo il funzionamento dell'interfacciamento tra CORSIM e strumenti esterni (CORSIM RTE) più dettagliatamente perché ci interessano al fine di ...
% presentiamo le API fornite dall'ambiente di sviluppo di TSIS
% mettere immagine di pulsante sensors in TSIS? far vedere come si aggiunge una RTE all'ambiente di TSIS (piccola sezione)

\subsubsection{CORWIN}

\subsubsection{CORSIM API}

\section{Estensione}
% TODO: introduzione, problema che deve risolvere

\subsection{Analisi}
% TODO: analisi del problema e del software che si è realizzato

\subsection{Sensors DLL}
% TODO: presentazione della DLL, progetto del software?

\section{Applicativi di supporto}
% TODO: discussione sulle operazioni utili alla creazione di dataset: insieme di file, ognuno dei quali è un insieme di traiettorie multivariate
