% !TEX encoding = UTF-8
% !TEX TS-program = pdflatex
% !TEX root = ../arsclassica.tex
% !TEX spellcheck = it-IT

%************************************************
\chapter{Creazione di dataset relativi al traffico}
\label{cap:tsis-sensors}
%************************************************
Al fine di valutare le prestazioni degli algoritmi di apprendimento e classificazione delle \acl{CTBN} è emersa la necessità di generare dei dataset adeguati a tale scopo.

Come già specificato in precedenza, le \acs{CTBN} sono un modello stocastico dedito alla rappresentazione dell'evoluzione di sistemi dinamici, cioè difenomeni cangianti nel tempo (non necessariamente o esclusivamente in base ad esso) rappresentabili come insieme di traiettorie multi-variate. Si è quindi optato di generare dei dataset che rappresentassero un tipico sistema cangiante: il traffico automobilistico su rete stradale.

Tali dataset, costituiti da un insieme di documenti rappresentanti la presenza (durante l'evolvere del tempo) di veicoli sui sensori di una rete stradale, sono stati generati con l'ausilio di un software commerciale, \acf{TSIS}, e di una sua \acl{RTE} appositamente sviluppata al fine di monitorare e tracciare il passaggio dei veicoli. 

In questo capitolo si presentano perciò i succitati strumenti utilizzati per la creazione di dataset relativi al traffico.

Relativamente, invece, al processo pratico di creazione dei dataset tramite gli strumenti discussi di seguito si rimanda alla~\autoref{sec:create-dataset-howto}.

\section{TSIS}
% TODO: introduzione a TSIS-CORSIM

\acf{TSIS} è un \acl{IDE}\footnote{Un \acl{IDE}, comunemente chiamato anche \ACF{IDE}, è un insieme di programmi finalizzati a supportare il processo di sviluppo dei software. Generalmente, un \acs{IDE} è costituito da uno strumento per la creazione e modifica del codice sorgente, un compilatore o un interprete, strumenti per l'automazione dello sviluppo e la qualità del codice sorgente.}, distribuito commercialmente da McTrans\footnote{\url{http://mctrans.ce.ufl.edu}} e supportato dalla \acf{FWHA}\footnote{Agenzia del Dipartimento dei Trasporti degli Stati Uniti d'America: \\ \url{www.fhwa.dot.gov}}, il cui scopo è permettere la simulazione e l'analisi di modelli di traffico stradale. \acs{TSIS} è costituito da insieme di strumenti dedicati alla creazione di reti stradali e relativi modelli di traffico, all'esecuzione di modelli di simulazione e all'interpretazione dei risultati di tali modelli. Esso è realizzato tramite un'architettura modulare, affinché l'utente possa, in caso di necessità, estendere tale ambiente creando degli ulteriori strumenti.

---
This guide describes how to install and begin using the Traffic Software Integrated System (TSIS). This guide is
intended to support traffic engineers using TSIS to conduct traffic operations analysis. However, it describes neither
the technical aspects of traffic simulations, nor the types of analyses that can be performed using traffic simulations.

The TSIS environment's graphical user interface (GUI) provides you with the ability to effectively manage your
traffic analysis projects and traffic analysis tools. This intuitive, user-friendly interface integrates those traffic
analysis tools in one environment so that you may easily access and apply them. On-line help for each tool is also
integrated into the TSIS interface.
For clarification, we introduce the following terminology. A TSIS project is a set of simulation cases that reflect a
common theme, e.g., signal timing variations for an artery in downtown Washington, D.C. A simulation case is a
single simulation for a specified traffic network as defined by its simulation input file, e.g., one of the signal timing
variations. A case includes the simulation input file and all data files generated by the simulation during a run.
Multiple runs of the simulation for gathering statistics is still considered part of a single case provided the input
(other than random number seeds) has not changed.
This version of TSIS includes several pre-configured tools including a GUI-base network and simulation input
editor (TRAFED), the CORSIM traffic simulation, and the TRAFVU animation and simulation analysis tool. These
and the other pre-configured components are described in more detail in the section titled, TSIS Package.

\subsection{Componenti}

\subsubsection{TShell}

TShell è la \acs{GUI} per 

is the graphical user interface for the TSIS integrated development environment. It provides a Project view
that enables you to manage your TSIS projects. It is also the container for the pre-configured tools and any tools
that you add to the suite. See the TShell User's Guide for additional details.

\subsubsection{TSIS Next}

TSIS Next contains the same type of functionality that can be seen in the TShell, TRAFED, and TRAFVU
component programs. TSIS Next is a “quicker-and-easier” version of TSIS that contains specific advantages and
disadvantages. Many CORSIM projects will continue to require TShell, TRAFED, and TRAFVU. By having
access to both TSIS and TSIS Next on the same computer, you can choose whichever functionality you prefer.

\subsubsection{CORSIM}

\acf{CORSIM} costituisce il componente principale, finalizzato alla modellazione e alla simulazione di modelli di traffico, dell'insieme di strumenti denominato \acs{TSIS}.

The CORSIM simulation consists of an integrated set of two microscopic simulation models (NETSIM and
FRESIM) that represent the entire traffic environment as a function of time. NETSIM represents surface-street
traffic and FRESIM represents freeway traffic. Microscopic simulations model the movements of individual
vehicles, which include the influences of driver behavior. Thus, the effects of very detailed strategies, such as
relocating bus stations or changing parking restrictions, can be studied with such models. CORSIM provides its
own interface in TSIS 6 that enables you to control the simulation and the accumulation of traffic measures of
effectiveness. See the CORSIM User's Guide for additional details.

\subsubsection{TRAFED}

TRAFED is a graphical user interface-based editor that allows you to easily create and edit traffic networks and
simulation input for the CORSIM model. See the TRAFED User's Guide for additional details.

\subsubsection{TRAFVU}

TRAFVU (TRAF Visualization Utility) is a graphics post-processor for FHWA’s CORSIM microscopic traffic
simulation system. TRAFVU displays traffic networks, animates simulated traffic flow operations, animates and displays simulation output measures of effectiveness, and displays user-specified input parameters for simulated
network objects. See the TRAFVU User's Guide for additional details.

\subsubsection{TSIS Text Editor}

This editor is a standard text editor that has the additional capability of understanding the CORSIM TRF file
format. When editing a TRF file with this editor, the TShell output window displays text describing the entry field
and record type at the current cursor position. Clicking a specific field description in the output window highlights
the corresponding entry field in the displayed TRF file. This makes manual editing of the text file much easier than
with previous text editors. See the TSIS Text Editor User's Guide for additional details.

\subsubsection{TSIS Script Tool}

The TSIS Script Tool is a combined script editor and tool for executing Visual Basic Scripts. Using the built-in
TSIS interfaces, the Script Tool is a powerful mechanism for extending the functionality of the other TSIS
components. We have also included two scripts with this release. One is a multi-run script that repeatedly runs
CORSIM on a test case, applying different random number seeds to each run. The other script runs CORSIM on
many different test cases. See the Script Tool User's Guide for additional details.

\subsubsection{TSIS Translator}

The TSIS Translator converts TRF files for use by TRAFED. This translator also performs the reverse operation of
translating the TRAFED native format (TNO) files into TRF files for use by CORSIM and other tools. See the
Translator User's Guide for additional details.

\subsubsection{TSIS Output Processor}

The TSIS Output Processor enables the user to automatically compute selected statistics and summary data during
multiple runs of CORSIM. The collected data is written to an Excel workbook, a comma-separated file, an XMLtagged
file, or a tab-separated text file. The Output Processor can also compute 95th percentile confidence intervals,
and can recommend sample sizes (i.e., the number of simulation runs that should be performed with varying random
number seeds) for achieving desired accuracy. The Output Processor has been redesigned for TSIS 6 to efficiently
summarize any model result generated by CORSIM. Cumulative MOEs may be obtained from the start of
simulation, or just for the current time interval, or just for the current time period, or any combination of those three.

\subsubsection{CORSIM Runtime Extension}

Although it comes pre-configured with a set of tools, TSIS provides a mechanism by which an external application
can interface directly with CORSIM simulation. This type of application has become known as a CORSIM run-time
extension (RTE). Run-time extensions can be built to replace existing logic in CORSIM, or to supplement the
logic. The original run-time extensions were tailored for signal timing studies. However, the concept has been
expanded to support freeway monitoring, incident detection and ramp metering run-time extension packages.

\subsection{Descrizione}
% TODO: descrizione di cosa

\subsection{API}
% TODO: descrizione delle varie API fornite da TSIS

\section{Estensione}
% TODO: introduzione, problema che deve risolvere

\subsection{Analisi}
% TODO: analisi del problema e del software che si è realizzato

\subsection{Sensors DLL}
% TODO: presentazione della DLL, progetto del software?

\section{Applicativi di supporto}
% TODO: discussione sulle operazioni utili alla creazione di dataset: insieme di file, ognuno dei quali è un insieme di traiettorie multivariate
