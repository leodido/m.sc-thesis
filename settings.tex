% !TEX encoding = UTF-8 Unicode
% !TEX TS-program = pdflatex
% !TEX root = arsclassica.tex
% !TEX spellcheck = it-IT

%*********************************************************************************
% Preamble
%*********************************************************************************

\sisetup{output-decimal-marker={,}}

%*********************************************************************************
% TOC settings
%*********************************************************************************
% \setcounter{secnumdepth}{5}
% \setcounter{tocdepth}{5}

%*********************************************************************************
% Chapters style
%*********************************************************************************
% Chapter number style: decomment if you want it with the same corpus font
% \renewcommand{\chapterNumber}{%
% \fontsize{70}{70}\usefont{\encodingdefault}{\sfdefault}{b}{n}%
% }%

% Workaround: multi-line titles
\renewcommand\formatchapter[1]{%
\begin{minipage}[b]{0.15\linewidth}
\chapterNumber
\end{minipage}%
\begin{minipage}[b]{0.70\linewidth}%length of the second row
\raggedright\spacedallcaps{#1}
\end{minipage}
}

%*********************************************************************************
% Impostazioni di amsmath, amssymb, amsthm
%*********************************************************************************

\DeclareMathAlphabet{\mathscrbf}{OMS}{mdugm}{b}{n}
\newcommand{\myinitial}{\mathscrbf{L}}

% Force the equation numbers to be always the same size
\makeatletter
\renewcommand{\maketag@@@}[1]{\hbox{\m@th\normalsize\normalfont#1}}%
\makeatother

% operators
\DeclareMathOperator*{\argmax}{arg\,max}
\DeclareMathOperator*{\argmin}{arg\,min}

% comandi per gli insiemi numerici (serve il pacchetto amssymb)
\newcommand{\numberset}{\mathbb}
\newcommand{\N}{\numberset{N}}
\newcommand{\R}{\numberset{R}}

% comandi per gli insiemi
\newcommand{\set}[1]{\mathbf{#1}}
\newcommand{\setel}[1]{\mathnormal{#1}}  % o \mathrm o \mathit

% comandi per i vettori
\newcommand{\vect}[1]{\set{#1}}
\newcommand{\vectel}[1]{\mathrm{#1}}

% comandi per i simboli matematici che denotano schemi, categorie, concetti
\newcommand{\conceptsym}[1]{\mathcal{#1}} % or use pazocal % try it

% un ambiente per i sistemi
\newenvironment{sistema}%
  {\left\lbrace\begin{array}{@{}l@{}}}%
  {\end{array}\right.}

% comandi per le lettere greche
% ...

% reference for equation terms. e.g., \tag*{\myterm{termA}}
\newcounter{term}
\renewcommand*{\theterm}{(\emph{\alph{term}})}
\AtBeginDocument{%
  \let\mylabel\label
}
\newcommand{\myterm}[1]{%
  \begingroup % keep the effects of \refstepcounter local
    \refstepcounter{term}%
    \mylabel{#1}%
    \text{\theterm}%
  \endgroup
}

%*********************************************************************************
% Nuovi ambienti: definizioni, teoremi etc. etc.
%*********************************************************************************
% definizioni (serve il pacchetto amsthm)
\newtheoremstyle{classicdef}% Nome
{12pt}% Spazio che precede l�enunciato
{12pt}% Spazio che segue l�enunciato
{}% Stile del font dell�enunciato
{}% Rientro (se vuoto, non c�� rientro,
% \parindent = rientro dei capoversi)
{\scshape}% Stile del font dell�intestazione
{:}% Punteggiatura che segue l�intestazione
{.5em}% Spazio che segue l�intestazione:
% " " = normale spazio inter-parola;
% \newline = a capo
{}% Specifica l�intestazione dell�enunciato
% (normalmente viene lasciata vuota)

\theoremstyle{definition}
\newtheorem{definizione}{Definizione}[chapter]
% \newtheorem{osservazione}[definizione]{Osservazione}
% \newtheorem{definizione}{Definizione}[section]
\newtheorem{osservazione}{Osservazione}[definizione]

% esempi
\theoremstyle{definition}
\newtheorem{esempio}{Esempio}[definizione]

% note about definitions
\theoremstyle{remark}
\newtheorem{notabene}{Nota}[definizione]

% note of section
\theoremstyle{remark}
\newtheorem{nota}{Nota}[section]

\theoremstyle{remark}
\newtheorem{notas}{Nota}[subsection]

% teoremi (serve il pacchetto amsthm)
\newtheoremstyle{classicthm}% Nome
{12pt}% Spazio che precede l�enunciato
{12pt}% Spazio che segue l�enunciato
{\itshape}% Stile del font dell�enunciato
{}% Rientro (se vuoto, non c�� rientro,
% \parindent = rientro dei capoversi)
{\scshape}% Stile del font dell�intestazione
{:}% Punteggiatura che segue l�intestazione
{.5em}% Spazio che segue l�intestazione:
% " " = normale spazio inter-parola;
% \newline = a capo
{}% Specifica l�intestazione dell�enunciato
% (normalmente viene lasciata vuota)

\theoremstyle{plain}
\newtheorem{teorema}{Teorema}[chapter]
\newtheorem*{teorema*}{Teorema}
\newtheorem{cor}[teorema]{Corollario}
\newtheorem{lem}[teorema]{Lemma}
\newtheorem{prop}[teorema]{Proposizione}
\newtheorem{oss}[teorema]{Osservazione}

% leggi (serve il pacchetto amsthm)
\newtheorem{legge}{Legge}
\newtheorem{decreto}[legge]{Decreto}

%*********************************************************************************
% Impostazioni di acronym
%*********************************************************************************
\newcommand{\acroname}{Acronimi}
% \renewcommand*{\acsfont}[1]{\textssc{#1}}                 % for MinionPro
\renewcommand*{\acsfont}[1]{\textsmaller{#1}}               % customize font for long version acronyms [works only if footnote not activate]
\renewcommand*{\acffont}[1]{#1}                             % idem, but for short version of acronyms
\renewcommand{\bflabel}[1]{{#1}\hfill}                      % fix the list of acronyms
\makeatletter                                               % macro that tweeks acronym package to render lowercase or not
\newif\if@in@acrolist
\AtBeginEnvironment{acronym}{\@in@acrolisttrue}
\newrobustcmd{\ul}[2]{\if@in@acrolist#1\else#2\fi}          % \ul{C}{c}iao defines an uppercase and lowercase variant of the same acronym
\newcommand{\ACF}[1]{{\@in@acrolisttrue\acf{#1}}}           % \ACF{<acronym>} force the defined capitalized variants of acronym letters/words
\newcommand{\ACL}[1]{{\@in@acrolisttrue\acl{#1}}}           % \ACL{<acronym}
\makeatother

%*********************************************************************************
% Impostazioni di biblatex
%*********************************************************************************
\defbibheading{bibliography}{%
\cleardoublepage
\manualmark
\phantomsection
\addcontentsline{toc}{chapter}{\tocEntry{\bibname}}
\chapter*{\bibname\markboth{\spacedlowsmallcaps{\bibname}}
{\spacedlowsmallcaps{\bibname}}}}

%*********************************************************************************
% Impostazioni di listings
%*********************************************************************************

%\lstloadlanguages{[Visual]C++,[ISO]C++}% try it

% Input source code
\makeatletter
\let\oldaddcontentsline\addcontentsline
\newcommand{\lstinputsourcecode}[2][]{{%
  \renewcommand{\lstlistingname}{Sorgente}%
  \renewcommand{\addcontentsline}[3]{\oldaddcontentsline{los}{##2}{##3}}%
  \lstinputlisting[#1]{#2}%
}}
\newcommand{\lstlistsourcecodename}{Elenco dei codici}%
\lst@UserCommand\lstlistofsourcecode{\bgroup
    \let\contentsname\lstlistsourcecodename
    \let\lst@temp\@starttoc \def\@starttoc##1{\lst@temp{los}}
    \tableofcontents \egroup}%
\makeatother

% Dichiarazioni utili per l'indice della tesi e le etichette
\providecommand*{\lstlistingautorefname}{algoritmo}
\providecommand*{\lstnumberautorefname}{linea}
%\addto\captionsitalian{%
%\renewcommand{\lstlistingname}{Algoritmo}}
%\addto\captionsitalian{%
%\renewcommand{\lstlistlistingname}{Elenco degli algoritmi}}

\definecolor{dkgreen}{rgb}{0,0.6,0}
\lstdefinelanguage{pseudo}{
    morekeywords=%
    {
        for,foreach,return,if,else,ifelse,max,min,in,len,log,index,var,function,seqlen,as,new,while,unique,which,inf,neighbors,score,null,break
    },
    morecomment=[l][\color{dkgreen}\ttfamily]{//},
    morecomment=[s][\color{dkgreen}\ttfamily]{/*}{*/},
    morestring=[b][\color{red}\rmfamily]",
    moredelim=[is][\color{orange}]{|*}{*|}
}
\lstset{
    language=pseudo,
    basicstyle=\normalsize\ttfamily,
    numbers=left,%none
    numberstyle=\scriptsize,%\tiny
    stepnumber=1,
    numbersep=8pt,
    showstringspaces=false,
    breaklines=true,
    frameround=ftff,%fttt,%
    frame=TB,%lines,
    tabsize=1,
    captionpos=b,
    escapeinside={<ls>}{<le>}
}
\lstdefinelanguage{cpp}{
    language=C++,
    basicstyle=\normalsize\ttfamily,
    breakatwhitespace=false,
    breaklines=true,
    captionpos=b,
    commentstyle=\color{dkgreen},
    deletekeywords={...},
    escapeinside={<ls>}{<le>},
    keywordstyle=\color{blue},
    morekeywords={__stdcall,__declspec,dllimport,dllexport},
    identifierstyle=\color{black},
    stringstyle=\color{red},
    numbers=left,
    numbersep=8pt,
    numberstyle=\scriptsize,
    rulecolor=\color{black},
    showspaces=false,
    showstringspaces=false,
    showtabs=false,
    stepnumber=1,
    tabsize=1,
    frameround=ftff,
    frame=TB,
    %title=\lstname
    %morecomment=[l][\color{magenta}]{\#}
}

%*********************************************************************************
% Impostazioni di hyperref
%*********************************************************************************
\hypersetup{%
    %hyperindex=true,
    %draft, % = elimina tutti i link (utile per stampe in bianco e nero)
    colorlinks=true, linktocpage=true, pdfstartpage=1, pdfstartview=FitV,%
    % decommenta la riga seguente per avere link in nero (per esempio per la stampa in bianco e nero)
    % colorlinks=false, linktocpage=false, pdfborder={0 0 0}, pdfstartpage=1, pdfstartview=FitV,%
    breaklinks=true, pdfpagemode=UseNone, pageanchor=true, pdfpagemode=UseOutlines,%
    plainpages=false, bookmarksnumbered, bookmarksopen=true, bookmarksopenlevel=1,%
    hypertexnames=true, pdfhighlight=/O,%nesting=true,%frenchlinks,%
    urlcolor=webbrown, linkcolor=RoyalBlue, citecolor=webgreen, %pagecolor=RoyalBlue,%
    urlcolor=Black, linkcolor=Black, citecolor=Black, %pagecolor=Black,%
    pdftitle={\mytitle},%
    pdfauthor={\textcopyright\ \myname, \myuni, \myfaculty},%
    pdfsubject={\mysubject},%
    pdfkeywords={\mykeywords},%
    pdfcreator={pdfLaTeX},%
    pdfproducer={LaTeX with hyperref, ClassicThesis and ArsClassica}%
}

% macro per far s� che solo la sezione e il numero di pagina siano un link in varioref
\makeatletter
\AtBeginDocument{
    \def\vr@f#1{%
      \leavevmode\unskip\vref@space
      \begingroup
        \hyperref[{#1}]{\ref*{#1}}%
          \vpageref[\unskip]{#1}%
      \endgroup
    }
}
\makeatother

%*********************************************************************************
% Analytical index
%*********************************************************************************

% shortcut for simple indexed terms
\newcommand{\keyword}[1]{#1\index{#1}}

% redefine lettergroup style
%\long\def\lettergroup#1\item{\item\textbf}
%\let\lettergroupDefault\lettergroup
%\makeatletter

\makeindex[program=texindy,options=-L italian -M arsclassica.xdy]

%*********************************************************************************
% Impostazioni relative alle immagini
%*********************************************************************************
\graphicspath{{images/}} % cartella dove sono riposte le immagini

\makeatletter\def\capfigure{\def\@captype{figure}}\makeatother

\makeatletter
\providecommand\phantomcaption{\caption@refstepcounter\@captype}
\makeatother

%*********************************************************************************
% A4 optimized margins
%*********************************************************************************
\areaset[current]{336pt}{750pt}
\setlength{\marginparwidth}{7em}
\setlength{\marginparsep}{2em}%

%*********************************************************************************
% Utilities
%*********************************************************************************
% make first letter uppercase
\makeatletter
\def\upcase{\expandafter\makeupcase}
\def\makeupcase#1{\uppercase{#1}}
\makeatother

% make first letter lowercase
\makeatletter
\def\lwcase{\expandafter\makelwcase}
\def\makelwcase#1{\lowercase{#1}}
\makeatother

%*********************************************************************************
% References for autoref
%*********************************************************************************
\addto\extrasitalian{%
  \def\figureautorefname{figura}%
}
\addto\extrasitalian{%
  \def\tableautorefname{tabella}%
}
\addto\extrasitalian{%
  \def\equationautorefname{equazione}%
}
\addto\extrasitalian{%
  \def\chapterautorefname{capitolo}%
}
\addto\extrasitalian{%
  \def\sectionautorefname{sezione}%
}
\addto\extrasitalian{%
  \def\subsectionautorefname{sottosezione}%
}
\addto\extrasitalian{%
  \def\subsubsectionautorefname{sottosezione}%
}
\addto\captionsitalian{%
    \renewcommand{\lstlistingname}{Algoritmo}%
}
\addto\captionsitalian{%
    \renewcommand{\lstlistlistingname}{Elenco degli algoritmi}%
}
\providecommand*{\lstlistingautorefname}{algoritmo}
\providecommand*{\lstnumberautorefname}{linea}

%*********************************************************************************
% Custom references for autoref
%*********************************************************************************
\newcommand{\definizioneautorefname}{definizione}
\newcommand{\osservazioneautorefname}{osservazione}
\newcommand{\notabeneautorefname}{nota}
\newcommand{\esempioautorefname}{esempio}
\newcommand{\termautorefname}{termine}
\newcommand{\teoremaautorefname}{teorema}

%*********************************************************************************
% Custom references for varioref
%*********************************************************************************
\makeatletter
\vref@addto\extrasitalian{\def\reftextfaraway#1{a pagina~\pageref{#1}}}
\makeatother

%*********************************************************************************
% Custom references for cref
%*********************************************************************************
\crefname{definizione}{definizione}{definizioni}
\crefname{lstlisting}{algoritmo}{algoritmi}

%*********************************************************************************
% Set footnote's continuos numbering and linking capabilities (using cleveref and chngcntr packages)
%*********************************************************************************
\counterwithout{footnote}{chapter}
\crefformat{footnote}{#2\footnotemark[#1]#3}

%*********************************************************************************
% Table customizations
%*********************************************************************************

% command for line below the heading of the tables has always thickness equal to \toprule and \bottomrule but it is vertically centered with respect to the row above and below (which is typical of the lines \midrule)
\newcommand{\otoprule}{\midrule[\heavyrulewidth]}

% new columns types: add + on the left of first column defition and ^ on the right of every other column definition (except the last one)
% insert \rowstyle command at the beginning of a row to format its cells. e.g. \rowstyle{\bfseries}
\newcolumntype{+}{>{\global\let\currentrowstyle\relax}}
\newcolumntype{^}{>{\currentrowstyle}}
\newcommand{\rowstyle}[1]{\gdef\currentrowstyle{#1}%
#1\ignorespaces
}

%*********************************************************************************
% Style custom commands
%*******************************************************************
\DeclareRobustCommand*{\pkgname}[1]{{\normalfont\sffamily#1}%
\index{Package!#1@\textsf{#1}}%
\index{#1@\textsf{#1}}}

\DeclareRobustCommand*{\envname}[1]{{\normalfont\ttfamily#1}%
\index{Environment!#1@\texttt{#1}}%
\index{#1@\texttt{#1}}}

\DeclareRobustCommand*{\optname}[1]{{\normalfont\ttfamily#1}%
\index{Option!#1@\texttt{#1}}%
\index{#1@\texttt{#1}}}

\DeclareRobustCommand*{\clsname}[1]{{\normalfont\sffamily#1}%
\index{Class!#1@\textsf{#1}}%
\index{#1@\textsf{#1}}}

\DeclareRobustCommand*{\cmdname}[1]{\mbox{\lstinline!\\#1!}%
\index{#1@\texttt{\hspace*{-1.2ex}\textbackslash#1}}}

\DeclareRobustCommand*{\classicthesis}{Classic\-Thesis}

\DeclareRobustCommand*{\arsclassica}{{\normalfont\sffamily ArsClassica}}

\DeclareRobustCommand*{\miktex}{MiK\TeX%
\index{miktex@MiK\protect\TeX}}

\DeclareRobustCommand*{\texlive}{\TeX{}~Live%
\index{texlive@\protect\TeX{}~Live}}

\DeclareRobustCommand*{\pacchetto}[1]{{\normalfont\ttfamily#1}%
\index{Pacchetto!#1@\texttt{#1}}%
\index{#1@\texttt{#1}}}

\DeclareRobustCommand*{\bibtex}{\textsc{Bib}\TeX%
\index{bibtex@\textsc{Bib}\protect\TeX}%
}

\DeclareRobustCommand*{\amseuler}{\protect\AmS{} Euler%
\index{AmS Euler@\protect\AmS~Euler}%
\index{Font!AmS Euler@\protect\AmS~Euler}}


%*********************************************************************************
% Define your commands here
%*********************************************************************************
% ...
\newcommand{\omissis}{\dots\negthinspace}
\newcommand{\virgolette}[1]{``#1''}
\newcommand{\caporali}[1]{�#1�}
\newcommand{\ie}{i.\,e.,}
\newcommand{\Ie}{I.\,e.,}
\newcommand{\eg}{e.\,g.,}
\newcommand{\Eg}{E.\,g.,}

% comuni
\newcommand{\ct}{continuos time}
\newcommand{\nb}{Naive Bayes}
\newcommand{\bn}{Bayesian Network}
\newcommand{\class}{classifier}
\newcommand{\ctbn}{\upcase \ct{} \bn{}}
\newcommand{\ctnb}{\upcase \ct{} \nb{}}
\newcommand{\cttanb}{\upcase \ct{} tree augumented \nb{}}
\newcommand{\pv}{variabile di processo}
\newcommand{\hc}{hill climbing}
\newcommand{\trs}{training set}
\newcommand{\tes}{test set}
\newcommand{\stats}{statistiche sufficienti}

\makeatletter
\newcommand{\omog}{omogene\@ifstar{o}{i}}
\makeatother

\makeatletter
\newcommand{\cond}{condizional\@ifstar{e}{i}}
\makeatother

\makeatletter
\newcommand{\im}{matric\@ifstar{e di intensit\`a}{i di intensit\`a}}
\makeatother

\makeatletter
\newcommand{\cim}{\@ifstar{\im*{} \cond*}{\im{} \cond}}
\makeatother

\makeatletter
\newcommand{\mprocess}{process\@ifstar{o di Markov}{i di Markov}}
\makeatother

\makeatletter
\newcommand{\ctmp}{\@ifstar{\mprocess*{} a tempo continuo}{\mprocess{} a tempo continuo}}
\makeatother

% elementi matematici comuni
\newcommand{\imsign}[1]{\set{Q}_{#1}}
\newcommand{\cimsign}[2]{\imsign{\,#1\,\arrowvert\,#2}}
\newcommand{\priorsign}[1]{\set{P}^0_{#1}}
\newcommand{\tstat}[1]{T[\,\setel{#1}\,]}
\newcommand{\mstat}[2][]{
    \ifstrempty{#1}{%
        M[\,\setel{#2}\,]%
    }{%
        M[\,\setel{#1}\,,\,#2\,]%
    }%
}
\newcommand{\param}[1]{\boldsymbol{\textbf{\text{#1}}}}
\newcommand{\paramhat}[1]{\boldsymbol{\hat{\textbf{#1}}}}
\newcommand{\greekhat}[1]{\boldsymbol{\skew{-.5}\hat#1}}
\newcommand{\evidencestream}[1]{(\set{x}^1\,,\,\set{x}^2\,,\,\dotsc\,,\,\set{x}^{#1})}

\newcommand\CC{C\nolinebreak[4]\hspace{-.05em}\raisebox{.4ex}{\relsize{-3}{\textbf{++}}}}

\newcommand{\mynum}[1]{\#\texorpdfstring{$#1$}{#1}}
\newcommand{\ds}[1]{dataset \mynum{#1}}
\newcommand{\Ds}[1]{Dataset \mynum{#1}}

%*********************************************************************************
% Hypenation exceptions
%*********************************************************************************
\hyphenation{Fortran ma-cro-istru-zio-ne nitro-idrossil-amminico tem-po-ra-li da-ta-set net-work ap-pren-di-men-to}


