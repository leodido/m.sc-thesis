% !TEX encoding = UTF-8
% !TEX TS-program = pdflatex
% !TEX root = ../arsclassica.tex
% !TEX spellcheck = it-IT

%*******************************************************
% Introduction
%*******************************************************
\cleardoublepage
\phantomsection
\pdfbookmark{Introduzione}{Introduzione}
\markboth{\spacedlowsmallcaps{Introduzione}}{\spacedlowsmallcaps{Introduzione}}
\addcontentsline{toc}{chapter}{\tocEntry{Introduzione}}
\chapter*{Introduzione}\label{cap:intro}
Il cambiamento è un concetto ubiquo.
Non è un'esagerazione pensare a tale concetto come l'unico aspetto costante della realtà.
A supporto di tale tesi, si pensi a un qualsiasi insieme di azioni reali.
Ad esempio si consideri l'investimento di denaro in azioni societarie, lo svolgimento di una partita di calcio, la cottura di un pasto o la cura di un paziente.
Qualsiasi sia il sistema dinamico che scegliamo di considerare, per comprenderlo e manipolarlo si rende necessario ragionare circa i cambiamenti che avvengono in esso.
Si ha infatti la necessità di prevedere con quale probabilità specifici eventi (che determinano il cambiamento dello stato del sistema) avvengono.
La natura di ogni cambiamento è spesso determinata da molti fattori che sono, a loro volta, cangianti.
Si ipotizzi, ad esempio, di voler predire il momento in cui un individuo troverà occupazione, ebbene tale evento è sicuramente influenzato dallo stato dell'economia locale in cui l'individuo vive, così come dalla sua stessa situazione finanziaria.
Allo stesso modo, per predire il tempo necessario affinché dei farmaci somministrati ad un paziente sortiscano il loro effetto è necessario considerare le tempistiche e le modalità con cui il paziente si è cibato.
\`E quindi evidente come nel concetto di cambiamento sia connaturato il concetto di tempo.
Interrogarsi circa l'avvenire dei cambiamenti nei sistemi citati, sebbene essi riguardino una vasta gamma di problemi, corrisponde ad analizzare le distribuzioni temporali degli eventi.

Per rispondere ad interrogativi di questo tipo, in letteratura si è solitamente ricorso all'utilizzo di modelli basati sui \mprocess{}
\citep{Duffie1996,Lando1998}.
Anche se tali approcci funzionano bene, essi hanno una limitazione: non permettono la specifica di modelli con un spazio degli stati strutturato in cui alcune variabili non dipendano direttamente da altre. Ne consegue che con tali approcci non è possibile modellare la distribuzione temporale che rappresenta la velocità d'effetto di un farmaco condizionatamente alla velocità con cui esso raggiunge il sangue del paziente, variabile la cui distribuzione nel tempo può essere a sua volta dipendente dalle tempistiche con cui si verificano altri fenomeni, come per esempio l'assunzione di cibo da parte del paziente.

Le \acf{BN}
\citep{Pearl1988}
permettono di rappresentare un dominio come uno spazio degli stati strutturato.
Esse codificano esplicitamente le dipendenze tra le variabili di un sistema e sfruttano le indipendenze al fine di semplificare la complessità computazionale del modello che lo descrive.
Tuttavia esse sono limitate alla rappresentazione di processi statici.
Da ciò consegue che non possono essere utilizzate per rispondere direttamente a interrogativi che riguardino i cambiamenti temporali di un sistema.

Le \acf{DBN} costituiscono l'estensione temporale delle \acl{BN}
\citep{Dean1989}
. Esse modellano la dinamica di un sistema discretizzando il tempo, suddividendolo così in un numero prefissato di intervalli temporali.
Per ognuno di tali intervalli temporali, le \acl{DBN} rappresentano tramite una \acl{BN} la distribuzione delle transizioni di stato che le variabili del sistema effettuano.
Nel concreto le \acf{DBN} fotografano lo stato di un sistema a differenti punti nel tempo, tutti equidistanti tra di essi.
Poiché tale strumento non rappresenta il tempo in modo esplicito risulta perciò difficile poterlo utilizzare per interrogarci circa l'occorrenza, in un qualsiasi momento temporale, di specifici eventi nel sistema che esso modella.
Una ulteriore limitazione di tale approccio è costituita dalla granularità temporale fissata a priori con cui le \acl{DBN} modellano l'evoluzione dei sistemi dinamici.
Nel caso in cui si intenda modellare un sistema i cui processi evolvono con differenti granularità temporali tramite le \acl{DBN}, è necessario rappresentare e vincolare l'intero sistema alla minore di tali granularità temporali.
Inoltre, nel caso in cui le osservazioni di cui si dispone avvengano a intervalli di tempo irregolari, il modello delle \acl{DBN} non prevede l'esclusione degli intervalli di tempo in cui non è stata ottenuta alcuna osservazione.

Tali premesse costituiscono il movente di questo lavoro di tesi, il cui obiettivo principale è presentare un framework alternativo, le \acf{CTBN}
\citep{Nodelman2007}
. La peculiarità di questo framework consiste nella sua abilità di rappresentare in modo esplicito l'evoluzione di sistemi dinamici nel tempo continuo, sorpassando così i vincoli dell'approccio utilizzato dalle \acl{DBN}.
Infatti, un modello \acl{CTBN} è in grado di adattare i parametri delle distribuzioni che rappresenta e la struttura di dipendenze che codifica alle granularità temporali relative all'evoluzione delle diverse variabili che lo compongono.
L'approccio che questo modello grafico probabilistico utilizza al fine di raggiungere tale scopo fonde idee provenienti dalle \acl{BN} e dalla teoria dei \mprocess{}.

Il presente elaborato è quindi composto dalla descrizione dei contributi sviluppati attorno alle \acl{CTBN}. Oltre alla descrizione di tale framework, finalizzato alla rappresentazione strutturata di \mprocess{} che evolvono nel tempo continuo, per il quale si è progettato e implementato un pacchetto \lstinline[]|R|, questo lavoro di tesi apporta principalmente i seguenti contributi:
\begin{itemize}
	\item descrizione, a partire dalle \acs{CTBN}, di una nuova \emph{classe di classificatori}, i \acf{CTBNC} \citep{Stella2012}
	\item descrizione del processo di \emph{classificazione superivisionata} di un classificatore \acl{CTBNC} (\acs{CTBNC}) e del processo di \emph{apprendimento strutturale} di una \acs{CTBN}; processi anch'essi implementati nel succitato pacchetto \lstinline[]|R|
	\item progettazione e implementazione di un software per il monitoraggio e \emph{tracciamento} del passaggio di \emph{veicoli} su reti stradali tramite sensori; sistema utile alla generazione di \emph{dataset} per il framework delle \acs{CTBN}
	\item applicazione del classificatore \acs{CTBNC} (\acs{CTBNC}) al problema della \emph{classificazione dei profili di traffico}.
\end{itemize}

Di seguito si passano in rassegna, capitolo per capitolo, gli argomenti affrontati.
\begin{description}
	\item[{\hyperref[cap:ctbn]{Nel primo capitolo}}]
	si introducono in primis i \emph{fondamenti teorici} su cui il framework delle \acf{CTBN} è basato. Successivamente si descrivono i \emph{concetti} e gli \emph{strumenti} da cui tale classe di modelli grafico probabilistici è costituita: \emph{\cim{}} e \emph{\stats{}}. Si affronta quindi il processo di \emph{apprendimento dei parametri} delle \acs{CTBN} da \emph{dati completi} e, infine, il calcolo della \emph{likelihood} di una \acs{CTBN} rispetto a un insieme di \emph{dati completi}.
	\item[{\hyperref[cap:ctbnc]{Nel secondo capitolo}}]
	si affronta il problema della \emph{classificazione supervisionata} di traiettorie multi-variate di variabili discrete a \emph{tempo continuo}. A tal scopo si descrive una nuova classe di modelli, la classe dei classificatori \acl{CTBN} (\acs{CTBNC}), derivata dalle \acs{CTBN}, e se ne presenta in particolare una sua specializzazione: i classificatori \acs{CTNB}. Viene illustrato sia un algoritmo generale per l'\emph{apprendimento dei parametri} dei classificatori \acs{CTBN}, sia un algoritmo per l'\emph{inferenza} della classe da associare a dei \emph{dati completi} dato a un modello di classificatore \acs{CTBN} (\acs{CTBNC}) precedentemente appreso.
	\item[{\hyperref[cap:structurallearning]{Nel terzo capitolo}}]
	si affronta il problema dell'\emph{apprendimento strutturale} di una \acs{CTBN} da \emph{dati completi}, per il quale si presenta un approccio risolutivo basato su punteggio. Viene presentata la definizione di una funzione che associa uno \emph{score bayesiano} ad ogni struttura rispetto a dei dati di addestramento. Infine viene descritto il funzionamento di una \emph{procedura di ottimizzazione} (\ie{} nello specifico si descrive l'algoritmo \emph{hill climbing}) finalizzata alla ricerca di una struttura che massimizzi la funzione di punteggio (\ie{} \emph{score bayesiano}).
	\item[{\hyperref[cap:ctbnr]{Nel quarto capitolo}}]
	si presentano le caratteristiche di \rctbn{}, il \emph{pacchetto} \lstinline[]|R| sviluppato al fine di implementare il framework delle \acs{CTBN} e gli algoritmi (\emph{apprendimento dei parametri} e \emph{apprendimento strutturale} di una \acs{CTBN}; \emph{classificazione supervisionata} di un \acs{CTBNC}) presentati nei precedenti capitoli. Inoltre, vengono presentati gli ulteriori moduli o pacchetti sviluppati con lo scopo di rendere maggiormente completo e snello l'intero processo di \emph{utilizzo del framework} delle \acs{CTBN}. Ci si riferisce, ad esempio, al pacchetto sviluppato per l'esecuzione di cross-validazione del succitato modello di classificazione supervisionata.
	\item[{\hyperref[cap:tsis-sensors]{Nel quinto capitolo}}]
	si introduce il problema di \emph{ottimizzazione del traffico urbano}. Successivamente si presentano le caratteristiche e il funzionamento di \acf{TSIS}, il sistema commerciale utilizzato per creare e simulare \emph{modelli di traffico}. Infine si descrive \acsfont{Sensors} \acs{DLL}, \acl{RTE} di \acs{TSIS} appositamente sviluppata al fine di monitorare e tracciare il passaggio dei veicoli sulle reti stradali tramite \emph{sensori}. Lo scopo di tale applicativo è la generazione di \emph{dataset} sottoponibili agli algoritmi di classificazione e apprendimento strutturale delle \acs{CTBN}. Tale passo è infatti propedeutico alla valutazione del processo di \emph{classificazione dei profili di traffico} tramite classificatori \acs{CTBN}.
	\item[{\hyperref[cap:esperimenti]{Nel sesto capitolo}}]
	si presentano le varie configurazioni dei $2$ \emph{modelli di traffico}, uno fittizio e uno che rispecchia una rete stradale reale (quella circostante \emph{Viale C. Battisti, $20900$ Monza, MB - Italia}), creati tramite \acs{TSIS} e simulati tramite il relativo simulatore, \acs{CORSIM}. Viene quindi fornita una descrizione dettagliata dei relativi \emph{dataset} generati dai succitati modelli di traffico tramite \acsfont{Sensors} \acs{DLL}. Per tutti i dataset generati, vengono presentati e comparati i risultati ottenuti utilizzando i classificatori \acs{CTBN} (\acs{CTBNC}) per la \emph{classificazione dei profili di traffico}.
	\item[{\hyperref[cap:concl]{Nel settimo capitolo}}]
	si riportano sia le \emph{conclusioni} tratte dalla sperimentazione attuata e riportata nel capitolo precedente, sia una serie di \emph{considerazioni} generali sui metodi e modelli presentati e implementati.
	\item[{\hyperref[cap:guide]{L'appendice A}}]
	offre le \emph{guide all'utilizzo} degli strumenti sviluppati a corredo di questo lavoro di tesi. In una prima parte viene presentato, tramite esempi, il funzionamento di \rctbn{}, il pacchetto \acsfont{R} che implementa quanto trattato del framework \acs{CTBN}. Nella seconda ed ultima parte viene invece presentata la generazione di dataset tramite l'esecuzione di \acsfont{Sensors} \acs{DLL} su modelli di traffico \acs{TSIS}.
\end{description}

