% !TEX encoding = UTF-8
% !TEX TS-program = pdflatex
% !TEX root = ../arsclassica.tex
% !TEX spellcheck = it-IT

%*******************************************************
% Introduction
%*******************************************************
\cleardoublepage
\pdfbookmark{Introduzione}{introduzione}
\chapter*{Introduzione}\label{cap:intro}

% \chapter{Introduzione}\label{cap:intro}

\lipsum[1]

Lorem ipsum dolor sit amet, consectetuer adipiscing elit.
% \begin{description}
% \item[{\hyperref[cap:lorem]{Il primo capitolo}}]
% offre una visione d'insieme della storia di \LaTeX{} e ne vengono presentate le idee di fondo.
% \item[{\hyperref[cap:ipsum]{Il secondo capitolo}}]
% spiega le operazioni, veramente semplici, per installare \LaTeX{} sul proprio calcolatore.
% \item[{\hyperref[cap:dolor]{L'appendice A}}] descrive  sinteticamente le principali norme tipografiche della lingua italiana, utili nella composizione di articoli, tesi o libri.
% \end{description}

\begin{description}
	\item[{\hyperref[cap:ctbn]{Nel primo capitolo}}]
	vengono dapprima introdotti i \emph{fondamenti teorici} su cui il framework delle \acf{CTBN} è basato. Successivamente si descrivono i \emph{concetti} e gli \emph{strumenti} da cui tale classe di modelli grafico probabilistici è costituita: \emph{\cim{}} e \emph{\stats{}}. Si affronta quindi il processo di \emph{apprendimento dei parametri} delle \acs{CTBN} da \emph{dati completi} e, infine, il calcolo della \emph{likelihood} di una \acs{CTBN} rispetto a un insieme di \emph{dati completi}.
	\item[{\hyperref[cap:ctbnc]{Nel secondo capitolo}}]
	viene affrontato il problema della \emph{classificazione supervisionata} di traiettorie multi-variate di variabili discrete a \emph{tempo continuo}. A tal scopo si descrive quindi una nuova classe di modelli, la classe dei \acf{CTBNC}, derivata dalla classe dei \acs{CTBN}. Per tale modello viene illustrato sia un algoritmo generale per l'\emph{apprendi\-mento dei parametri}, sia un algoritmo per l'\emph{inferenza} della classe associata a dei \emph{dati completi} rispetto al modello di un classificatore (\acs{CTNB}) \acs{CTBNC} precedentemente appreso.
	\item[{\hyperref[cap:structurallearning]{Nel terzo capitolo}}]
	si affronta il problema dell'\emph{apprendimento strutturale} di una \acs{CTBN} da \emph{dati completi}. Si presenta un approccio risolutivo basato su punteggio. Quindi, il capitolo prosegue spiegando le componenti di tale approccio. Viene presentata la definizione di una funzione che associa uno \emph{score bayesiano} ad ogni struttura rispetto ai dati di addestramento. Infine viene descritto il funzionamento di una \emph{procedura di ottimizzazione} (\ie{} nello specifico si descrive l'algoritmo \emph{hill climbing}) finalizzata alla ricerca di una struttura che massimizzi lo \emph{score bayesiano}.
	\item[{\hyperref[cap:ctbnr]{Nel quarto capitolo}}]
	\omissis{}
	%offre una descrizione dell'implementazione, attuata in linguaggio \acsfont{R}, degli algoritmi presentati nei precedenti capitoli
	%\item[{\hyperref[cap:tsis-sensors]{Nel quinto capitolo}}]
	%si introduce il problema di ottimizzazione del traffico urbano. Successivamente si presentano le caratteristiche e il funzionamento di \acf{TSIS}, il sistema commerciale utilizzato per creare e simulare modelli di traffico. Infine si descrive \acsfont{Sensors} \acs{DLL}, \acl{RTE} di \acs{TSIS} appositamente sviluppata al fine di monitorare e tracciare il passaggio dei veicoli sulle reti stradali tramite sensori. Lo scopo di tale applicativo è la generazione di dataset sottoponibili agli algoritmi di classificazione e apprendimento strutturale delle \acs{CTBN}.
	%\item[{\hyperref[cap:esperimenti]{Nel sesto capitolo}}]
	%si presentano le varie configurazioni dei $2$ modelli di traffico, uno fittizzio e uno che rispecchia una rete stradale reale (quella circostante \emph{Viale C. Battisti, $20900$ Monza, MB - Italia}), creati tramite \acs{TSIS} e simulati tramite il relativo simulatore, \acs{CORSIM}. Nel prosieguo segue una descrizione dei relativi dataset generati dai succitati modelli di traffico tramite \acsfont{Sensors} \acs{DLL}. Infine vengono presentati i risultati della classificazione dei profili di traffico ottenuti applicando il modello \acs{CTNBC} e dell'apprendimento strutturale.
	%\item[{\hyperref[cap:guide]{L'appendice A}}]
	%descrive sinteticamente le principali norme tipografiche della lingua italiana, utili nella composizione di articoli, tesi o libri.
\end{description}

