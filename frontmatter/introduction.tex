% !TEX encoding = UTF-8
% !TEX TS-program = pdflatex
% !TEX root = ../arsclassica.tex
% !TEX spellcheck = it-IT

%*******************************************************
% Introduction
%*******************************************************
\cleardoublepage
\pdfbookmark{Introduzione}{introduzione}
\chapter*{Introduzione}\label{cap:intro}

Di seguito si passano in rassegna gli argomenti trattati in ogni capitolo.
\begin{description}
	\item[{\hyperref[cap:ctbn]{Nel primo capitolo}}]
	si introducono in primis i \emph{fondamenti teorici} su cui il framework delle \acf{CTBN} è basato. Successivamente si descrivono i \emph{concetti} e gli \emph{strumenti} da cui tale classe di modelli grafico probabilistici è costituita: \emph{\cim{}} e \emph{\stats{}}. Si affronta quindi il processo di \emph{apprendimento dei parametri} delle \acs{CTBN} da \emph{dati completi} e, infine, il calcolo della \emph{likelihood} di una \acs{CTBN} rispetto a un insieme di \emph{dati completi}.
	\item[{\hyperref[cap:ctbnc]{Nel secondo capitolo}}]
	si affronta il problema della \emph{classificazione supervisionata} di traiettorie multi-variate di variabili discrete a \emph{tempo continuo}. A tal scopo si descrive una nuova classe di modelli, la classe dei classificatori \acl{CTBN} (\acs{CTBNC}), derivata dalle \acs{CTBN}, e se ne presenta in particolare una sua specializzazione: i classificatori \acs{CTNB}. Viene illustrato sia un algoritmo generale per l'\emph{apprendimento dei parametri} dei classificatori \acs{CTBN}, sia un algoritmo per l'\emph{inferenza} della classe da associare a dei \emph{dati completi} dato a un modello di classificatore \acs{CTNB} (\acs{CTNBC}) precedentemente appreso.
	\item[{\hyperref[cap:structurallearning]{Nel terzo capitolo}}]
	si affronta il problema dell'\emph{apprendimento strutturale} di una \acs{CTBN} da \emph{dati completi}, per il quale si presenta un approccio risolutivo basato su punteggio. Viene presentata la definizione di una funzione che associa uno \emph{score bayesiano} ad ogni struttura rispetto a dei dati di addestramento. Infine viene descritto il funzionamento di una \emph{procedura di ottimizzazione} (\ie{} nello specifico si descrive l'algoritmo \emph{hill climbing}) finalizzata alla ricerca di una struttura che massimizzi la funzione di punteggio (\ie{} \emph{score bayesiano}).
	\item[{\hyperref[cap:ctbnr]{Nel quarto capitolo}}]
	\omissis{}
	%offre una descrizione dell'implementazione, attuata in linguaggio \acsfont{R}, degli algoritmi presentati nei precedenti capitoli
	\item[{\hyperref[cap:tsis-sensors]{Nel quinto capitolo}}]
	si introduce il problema di ottimizzazione del traffico urbano. Successivamente si presentano le caratteristiche e il funzionamento di \acf{TSIS}, il sistema commerciale utilizzato per creare e simulare modelli di traffico. Infine si descrive \acsfont{Sensors} \acs{DLL}, \acl{RTE} di \acs{TSIS} appositamente sviluppata al fine di monitorare e tracciare il passaggio dei veicoli sulle reti stradali tramite sensori. Lo scopo di tale applicativo è la generazione di dataset sottoponibili agli algoritmi di classificazione e apprendimento strutturale delle \acs{CTBN}. Tale passo è infatti propedeutico alla valutazione del processo di \emph{classificazione dei profili di traffico} tramite classificatori \acs{CTBN}.
	\item[{\hyperref[cap:esperimenti]{Nel sesto capitolo}}]
	si presentano le varie configurazioni dei $2$ modelli di traffico, uno fittizzio e uno che rispecchia una rete stradale reale (quella circostante \emph{Viale C. Battisti, $20900$ Monza, MB - Italia}), creati tramite \acs{TSIS} e simulati tramite il relativo simulatore, \acs{CORSIM}. Viene quindi fornita una descrizione dettagliata dei relativi dataset generati dai succitati modelli di traffico tramite \acsfont{Sensors} \acs{DLL}. Per tutti i dataset generati, vengono presentati e commentati i risultati ottenuti utilizzando i classificatori \acs{CTNB} (\acs{CTNBC}) per la \emph{classificazione dei profili di traffico}. Si riportano, inoltre, i risultati dell'applicazione dell'algoritmo di \emph{apprendimento strutturale} dei \acs{CTBN} ai succitati dataset.
	\item[{\hyperref[cap:concl]{Nel settimo capitolo}}]
	\omissis{}
	% considerazioni/conclusioni
	% sviluppi futuri
	\item[{\hyperref[cap:guide]{L'appendice A}}]
	offre le \emph{guide all'utilizzo} degli strumenti sviluppati a corredo di questo lavoro di tesi. In una prima parte viene presentato, tramite esempi, il funzionamento di \acsfont{RCTBN}, il pacchetto \acsfont{R} che implementa quanto trattato del framework \acs{CTBN}. Nella seconda ed ultima parte viene invece presentata la generazione di dataset tramite l'esecuzione di \acsfont{Sensors} \acs{DLL} su modelli di traffico \acs{TSIS}.
\end{description}



