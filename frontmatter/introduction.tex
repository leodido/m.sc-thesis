% !TEX encoding = UTF-8
% !TEX TS-program = pdflatex
% !TEX root = ../arsclassica.tex
% !TEX spellcheck = it-IT

%*******************************************************
% Introduction
%*******************************************************
\cleardoublepage
\pdfbookmark{Introduzione}{introduzione}
\chapter*{Introduzione}\label{cap:intro}

%abstract 1
In this paper we present a language for finite state con- tinuous time Bayesian networks (CTBNs), which de- scribe structured stochastic processes that evolve over continuous time. The state of the system is decom- posed into a set of local variables whose values change over time. The dynamics of the system are described by specifying the behavior of each local variable as a function of its parents in a directed (possibly cyclic) graph. The model specifies, at any given point in time, the distribution over two aspects: when a local variable changes its value and the next value it takes. These distributions are determined by the variable’s current value and the current values of its parents in the graph. More formally, each variable is modelled as a finite state continuous time Markov process whose transi- tion intensities are functions of its parents. We present a probabilistic semantics for the language in terms of the generative model a CTBN defines over sequences of events. We list types of queries one might ask of a CTBN,discuss the conceptual and computational diffi- culties associated with exact inference, and provide an algorithm for approximate inference which takes ad- vantage of the structure within the process.
%intro 1
Consider a medical situation where you have administered a drug to a patient and wish to know how long it will take for the drug to take effect. The answer to this question will likely depend on various factors, such as how recently the patient has eaten. We want to model the temporal process for the effect of the drug and how its dynamics depend on these other factors. As another example, we might want to predict the amount of time that a person remains unem- ployed, which can depend on the state of the economy, on their own financial situation, and more.
Although these questions touch on a wide variety of is-
sues, they are all questions about distributions over time.

%abstract 2
The class of continuous time Bayesian network classifiers is defined; it solves the problem of supervised classification on multivariate trajectories evolving in continuous time. The trajectory consists of the val- ues of discrete attributes that are measured in continuous time, while the predicted class is expected to occur in the future. Two instances from this class, namely the continuous time naive Bayes classifier and the continuous time tree augmented naive Bayes classifier, are introduced and analyzed. They implement a trade-off between computational complexity and classification accuracy. Learning and inference for the class of continuous time Bayesian network classifiers are addressed, in the case where complete data are available. A learning algorithm for the continuous time naive Bayes classifier and an exact inference algo- rithm for the class of continuous time Bayesian network classifiers are described. The performance of the continuous time naive Bayes classifier is assessed in the case where real-time feedback to neurological patients undergoing motor rehabilitation must be provided.
%intro 2

Di seguito si passano in rassegna gli argomenti trattati in ogni capitolo.
\begin{description}
	\item[{\hyperref[cap:ctbn]{Nel primo capitolo}}]
	si introducono in primis i \emph{fondamenti teorici} su cui il framework delle \acf{CTBN} è basato. Successivamente si descrivono i \emph{concetti} e gli \emph{strumenti} da cui tale classe di modelli grafico probabilistici è costituita: \emph{\cim{}} e \emph{\stats{}}. Si affronta quindi il processo di \emph{apprendimento dei parametri} delle \acs{CTBN} da \emph{dati completi} e, infine, il calcolo della \emph{likelihood} di una \acs{CTBN} rispetto a un insieme di \emph{dati completi}.
	\item[{\hyperref[cap:ctbnc]{Nel secondo capitolo}}]
	si affronta il problema della \emph{classificazione supervisionata} di traiettorie multi-variate di variabili discrete a \emph{tempo continuo}. A tal scopo si descrive una nuova classe di modelli, la classe dei classificatori \acl{CTBN} (\acs{CTBNC}), derivata dalle \acs{CTBN}, e se ne presenta in particolare una sua specializzazione: i classificatori \acs{CTNB}. Viene illustrato sia un algoritmo generale per l'\emph{apprendimento dei parametri} dei classificatori \acs{CTBN}, sia un algoritmo per l'\emph{inferenza} della classe da associare a dei \emph{dati completi} dato a un modello di classificatore \acs{CTNB} (\acs{CTNBC}) precedentemente appreso.
	\item[{\hyperref[cap:structurallearning]{Nel terzo capitolo}}]
	si affronta il problema dell'\emph{apprendimento strutturale} di una \acs{CTBN} da \emph{dati completi}, per il quale si presenta un approccio risolutivo basato su punteggio. Viene presentata la definizione di una funzione che associa uno \emph{score bayesiano} ad ogni struttura rispetto a dei dati di addestramento. Infine viene descritto il funzionamento di una \emph{procedura di ottimizzazione} (\ie{} nello specifico si descrive l'algoritmo \emph{hill climbing}) finalizzata alla ricerca di una struttura che massimizzi la funzione di punteggio (\ie{} \emph{score bayesiano}).
	\item[{\hyperref[cap:ctbnr]{Nel quarto capitolo}}]
	\omissis{}
	%offre una descrizione dell'implementazione, attuata in linguaggio \acsfont{R}, degli algoritmi presentati nei precedenti capitoli
	\item[{\hyperref[cap:tsis-sensors]{Nel quinto capitolo}}]
	si introduce il problema di ottimizzazione del traffico urbano. Successivamente si presentano le caratteristiche e il funzionamento di \acf{TSIS}, il sistema commerciale utilizzato per creare e simulare modelli di traffico. Infine si descrive \acsfont{Sensors} \acs{DLL}, \acl{RTE} di \acs{TSIS} appositamente sviluppata al fine di monitorare e tracciare il passaggio dei veicoli sulle reti stradali tramite sensori. Lo scopo di tale applicativo è la generazione di dataset sottoponibili agli algoritmi di classificazione e apprendimento strutturale delle \acs{CTBN}. Tale passo è infatti propedeutico alla valutazione del processo di \emph{classificazione dei profili di traffico} tramite classificatori \acs{CTBN}.
	\item[{\hyperref[cap:esperimenti]{Nel sesto capitolo}}]
	si presentano le varie configurazioni dei $2$ modelli di traffico, uno fittizzio e uno che rispecchia una rete stradale reale (quella circostante \emph{Viale C. Battisti, $20900$ Monza, MB - Italia}), creati tramite \acs{TSIS} e simulati tramite il relativo simulatore, \acs{CORSIM}. Viene quindi fornita una descrizione dettagliata dei relativi dataset generati dai succitati modelli di traffico tramite \acsfont{Sensors} \acs{DLL}. Per tutti i dataset generati, vengono presentati e commentati i risultati ottenuti utilizzando i classificatori \acs{CTNB} (\acs{CTNBC}) per la \emph{classificazione dei profili di traffico}. Si riportano, inoltre, i risultati dell'applicazione dell'algoritmo di \emph{apprendimento strutturale} dei \acs{CTBN} ai succitati dataset.
	\item[{\hyperref[cap:concl]{Nel settimo capitolo}}]
	\omissis{}
	% considerazioni/conclusioni
	% sviluppi futuri
	\item[{\hyperref[cap:guide]{L'appendice A}}]
	offre le \emph{guide all'utilizzo} degli strumenti sviluppati a corredo di questo lavoro di tesi. In una prima parte viene presentato, tramite esempi, il funzionamento di \acsfont{RCTBN}, il pacchetto \acsfont{R} che implementa quanto trattato del framework \acs{CTBN}. Nella seconda ed ultima parte viene invece presentata la generazione di dataset tramite l'esecuzione di \acsfont{Sensors} \acs{DLL} su modelli di traffico \acs{TSIS}.
\end{description}



