% !TEX encoding = UTF-8
% !TEX TS-program = pdflatex
% !TEX root = ../arsclassica.tex
% !TEX spellcheck = it-IT

%*******************************************************
% Colophon
%*******************************************************
\clearpage
\phantomsection
\thispagestyle{empty}
\pdfbookmark{Colophon}{Colophon}

\hfill

\vfill

\noindent
Questa tesi è stata scritta utilizzando \ctLaTeX{}. Il modello tipografico che si è adottato è una personalizzazione dell'autore (reperibile all'indirizzo \url{https://github.com/leodido/arsclassica}) del modello offerto dal pacchetto \href{http://www.ctan.org/tex-archive/macros/latex/contrib/arsclassica}{\arsclassica{}}.
Si sono utilizzati i seguenti font:
\begin{itemize}
	\item \rmfamily{Palatino} per il testo
	\item \ttfamily{Euler}\rmfamily{} per le equazioni, i listati e i numeri
	\item \sffamily{Iwona}\rmfamily{} sia per le unità di sezionamento (capitoli, sezioni, sottosezioni) sia per le etichette degli elenchi di descrizioni e degli elementi fluttuanti.
\end{itemize}

\noindent
I grafici sono stati creati in linguaggio \lstinline$R$ \citep{R2013} con l'ausilio del pacchetto \lstinline$ggplot2$ \citep{GGPLOT2009}.

\vspace{.2cm}
\hrule
\bigskip

\noindent\myname: \textit{\mytitle}. \\
\mydegree \hspace{1px} \textcopyright\ \MakeTextLowercase{\mytime}. \\

\medskip
\noindent{\spacedlowsmallcaps{E-mail}}: \\
\mail{l.didonato@campus.unimib.it} \\
\mail{leodidonato@gmail.com}
